\documentclass[kn.tex]{subfiles}

\begin{document}

\noindent\begin{longtable}{ l p{12cm} }
\MSN & \emph{Nairātmyāprakāśa} by Advayavajra. In \emph{Hevajrasādhanasaṅgraha}, ff. 260r5–264v5. \\

\TIB & \emph{bdag med ma'i rab tu gsal ba} by gNyis su med pa'i rdo rje. Translation by Vajrapāṇi and Jñānākara (Ye shes 'byung gnas). In sDe dge bstan 'gyur, Tōh.\ 1308, rgyud, vol.\ 10 (\emph{ta}), ff.\ 218v5-223r1. \\

MuĀ & \emph{Muktāvalī} \\

HeTa & \emph{Hevajratantra} \bigskip \\

$ac$ & \emph{ante correctionem} \\
\conj & conjecture \\
D & sDe dge \\
\emph{deest} & ommitted in \\
\diag & diagnostic conjecture [e.g. `reconstructed' from Tibetan]\\
\emd & emendation \\
f./fo. & folio/folios \\
$pc$ & \emph{post correctionem} \\
$r$ & recto \\
$v$ & verso \\
$\Sigma$\textsubscript{X} & Reading is shared in all but witness X. \\
((kiṃcit)) & Reading is uncertain—either illegible or otherwise in doubt. \\
<kiṃcit> & Reading is cancelled. \\
\crux kiṁcit\crux & Reading does not make sense to the editor and an adequate conjecture was not able to be chosen. \\
{[}kiṃcit{]} & Indication of a diagnostic conjecture.  \\
\varbrace{l}{\llcorner}kiṃcit\varbrace{l}{\lrcorner} & Indication of a lemma. \\
$\perp $ & Change of folio/page. \\
	.. & Damaged \emph{akṣara} (one . per half \emph{akṣara}) \\
... & Lacunae of an unknown quanity of \emph{akṣara}s. \\
° & Mark of abbreviation. \\
\end{longtable}

\end{document}
