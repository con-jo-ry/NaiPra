\documentclass[naipra.tex]{subfiles}
\begin{document}

\noindent \emph{oṁ} Homage to Nairātmyā!

\begin{quote}
	May the \emph{bhagavān} Vajrasattva—whom they teach to be the \emph{dharmakāya}, devoid of conceptualisation (\emph{parikalpa}), [and] the attractive \emph{saṃbhogakāya} that is simply incomparable bliss; who has a \emph{nirmāṇakāya} because of his accomplishing what is beneficial for the world—lead to your prosperity/liberation (\emph{śreyas})!

	I will teach a Nairāmtyāsādhana, in accordance with my judgement (\emph{yathāmati}) and with scripture (\emph{yathāgama}), clearly laid out in one place, including some small details (\emph{vistara}) for the ignorant.
\end{quote}

Now, in a location such as a charnel ground that is mentally suitable, a \emph{yogin}, who consumes the pledge substances of the five nectars and so on, takes up a comfortable seat.
With neither attachment nor inhibition, and with his mind set on the aims of sentient beings, he should give rise to the pride of Nairātmyā and focus on a blue \emph{hūṁ} on the sun at his heart.
Then, drawing [him] in with light rays that were lit by that [syllable]—[light rays] which radiate through the triple world—[the \emph{yogin}] should visualise in front as residing in the sky [Heruka]:
He abides in the Akaniṣṭha heaven.
He is surrounded by the eight \emph{yoginī}s.
He has sixteen arms and eight faces.
His crowns are made of skull garlands.
He tramples the four Māras with his four feet;
He is dark blue in colour.
He has completely fused (\emph{sakalanilīna}) into the skulls held (\emph{kalita}) in his right group of hands (\emph{karanikara}) an elephant, horse, donkey, bull, camel, human, \emph{śarabha}, and cat.
He has residing in the lotus vessels situated in his left group of hands (\emph{pāṇikadamba}) the earth goddess, the water god, the wind god, the fire god, the moon (\emph{rajanīnātha}, the `lord of the night'), the sun, Yama, and Dhanada (i.e., Kubera).
His primary face is black.
His right face is white like the moon or jasmine. 
His left face is intensely red.
His dreadful upper face is dark grey.
All of his remaining faces are all deep black.
He is adorned with a garland of fifty skulls, and his neck is embraced by Nairātmyā. 

Immediately thereafter, the [\emph{yogin}] should have the eight \emph{yoginī}s worship [Heruka] with the outer, secret, and reality forms of worship. 
And here, Picuvajra [i.e., the eight-faced Heruka] is worshipped in order to realise that insight and means are identical in nature.
Then [the \emph{yogin}] should bow and confess his sins and vow not to commit them again.
He should rejoice in merit, embrace the triple refuge, give rise to the intention to achieve awakening (\emph{bodhicitta}), offer his body, and request [the turning of the wheel of Dharma].
Then, after meditating on the four supreme states (\emph{caturbrahmavihāra}), he should rest in a way such that he does not remain fixed [anywhere] (\emph{apratiṣṭhitarūpa}) while meditating on the meaning of the mantra which is by nature the drawing together of the core reality of all things—\begin{mantra}oṁ śūnyatājñānavajrasvabhāvātmako 'haṃ śūnyatājñānavajrasvabhāvātmakāḥ sarvadharmāḥ\end{mantra} (`oṁ, I am in essence the vajra nature of the knowledge of emptiness; all phenomena are in essence the vajra nature of knowledge of emptiness').

Then, recollecting his resolve (\emph{praṇidhi}), [the \emph{yogin}] should arise from his meditative concentration and visualise a sun disc in front [of himslef], created by the letter \emph{ra}.
And he should visualise, formed by the letter \emph{hūṁ} on that [sun disc], a crossed vajra.
Created by vajras densely packed with light which emerge from that crossed vajra, a vajra fence, a canopy (\emph{pañjara}) and a line of enclosure, a vajra-made ground below, and a mote should be visualised by the [\emph{yogin}].
And with the sun and vajra, which turn into light rays and spread everywhere, [the \emph{yogin}] should make firm all that [which has been visualised up to this point].

Immediately after that, [the \emph{yogin}] should visualise in space insight in the form of the \emph{dharmodaya/ā}: its interior is hollow (\emph{antaḥsuṣira}), it is extremely (\emph{atibahala}) white, and it is upright (\emph{ūrdhva}).
Then he should meditate on a broad, multi-coloured and eight-petaled lotus as present inside that [insight].
Then [the \emph{yogin}] should reflect on a \emph{ra} in the middle of that [lotus], from which a sun disc arises, in the middle of which a \emph{hūṁ} transforms into a crossed vajra. 
And in the middle of the crossed vajra, he should visualise as [stacked] above one another (\emph{upary upari}) the wind, fire, water, and earth [elements]—which are [\emph{maṇḍala}s that are] grey, red, white, and green; in the shapes of a bow, a triangle, a globe, and a square; and transforming out of the syllables \emph{yaṃ}, \emph{raṃ}, \emph{vaṃ}, and \emph{laṃ}.
Regarding (\emph{ā-√kal}) all of this as merely awareness (\emph{jñānamātra}), he should visualise as transformed out of those elements a temple palace (\emph{kūṭāgāra})—it has four corners and four doors, it is beautified with eight pillars, and it is ornamented with necklaces (\emph{hāra}) and half necklaces (\emph{ardhahāra}).

Then, on the interior of the fence, [the \emph{yogin}] should bring to mind the eight charnal grounds.
Here, in the east is Indra [and], in a yellow myrobalan tree (\emph{harītakīvṛkṣa}), the dark (\emph{mecaka}) elephant-faced [\emph{yakṣa}].
In the south is Yama [and], in a mango tree, the buffalo faced [\emph{yakṣa}] who is white.
In the west, at an ashoka tree, is Varuṇa (the water god) [and] the red lion-faced [\emph{yakṣa}].
In the north, at a bodhi tree, is Kubera [and] the green %
	% other sources appear to read yellow
man-faced [\emph{yakṣa}]. 

In the south east, at a \emph{karañja} tree (Indian beech tree), is Agni (Vaiśvānara) [and] the white goat-faced [\emph{yakṣa}]. % red in other sources
In the south west, at a fiddle-leaf fig tree, is Nairṛta/Nirṛti (Naravāhana) [and] the pale-white human-faced [\emph{yakṣa}]. % Black in other sources
In the north west, at an arjun tree, is Vāyu (Pavana) [and] the yellow deer-faced [\emph{yakṣa}]. % śyāma VaPra, dhūmra AŚma 
In the north east, there is Śiva (Bhūteśa) [and], in a banyan tree, the variegated bull-faced [\emph{yakṣa}]. % white in other sources
And all of these [\emph{yakṣa}s] have skull cups held in their left hands, have their right hands occupied with various weapons, and reveal the upper halves of their bodies.

Similarly, in the eight direction, in order beginning in the south, there are [the \emph{nāga}s] Ananta, Padma, Vāsuki, Mahāpadma, Takṣaka, Śaṃkhapāla, Karkoṭa, and Kulika.
And eight clouds should be brought to mind, which are dark (\emph{mecaka}), white (\emph{śukla}), white/black (\emph{śiti}), pale-white (\emph{pāṇḍu}), red, yellow, green. 
And similarly, the guardians of the directions are their well-known colours, and the eight \emph{caitya}s should be understood to be corresponding colours. 

Then [the \emph{yogin}] should call to mind a red lotus with eight flower in the middle of the \emph{maṇḍala} and then visualise fifteen corpses on the four petals of the east and so forth in the middle of that lotus, likewise at the four corners of the deity palace, at the four doors, and below and above.
Then he should meditated on the fifteen vowels, i.e.\ \emph{a} and so forth, which are mounted atop the corpses and situated in the middle of moons and suns that have transformed from the line of vowels and the line of consonants.
Then, by the transforming of \emph{a} and the other seed syllables, [there appears] in all places choppers (\emph{kartikā}) with the [corresponding] syllable.
Then, with the transforming of the moons, suns, and choppers with their seed syllables, he should meditate on the fifteen \emph{yoginī}s, beginning with Nairātmyā.
Of these [elements in the meditation], the moon has the mirror-like knowledge; the sun has the knowledge of equality; the seed syllable placed in between them is discriminating [knowledge]; the oneness of all of these [the knowledge of] the performance of activities; and the arising of the form is the fully pure Dharma Realm.

Here in the middle of the central part of the lotus (\emph{varaṭaka}) Nairāmtyā should be meditated on: she arises from the vowel \emph{a}; she is of the nature of enmity (\emph{dveśa});\footnote{
	That is to say, she belongs to the \emph{dveṣakula} of Akṣobhya.
} she is sealed by Akṣobhya; she is of the nature of the consciousness aggregate; she has wisdom and means as her essence; her neck is embraced by a \emph{khaṭvāṅga}, which takes the form of her external `means' [i.e., male consort].
On the petals in the east and so forth, [the \emph{yogin}] should visualise Vajrā, Gaurī, Vārī, and Vajraḍākinī: they arise from \emph{ā}, \emph{i}, \emph{ī}, and \emph{u}; they are of the nature of delusion (\emph{moha}), back-biting (\emph{paiśunya}), passion (\emph{rāga}), and jealousy (\emph{īrṣyā}); they are sealed by Vairocana, Ratnasambhava, Amitābha, and Amoghasiddhi; and they are of the nature of the aggregates of material form (\emph{rūpa}), sensation (\emph{vedanā}), identification (\emph{saṃjñā}), conditioning factors (\emph{saṃskāra}), and consciousness (\emph{vijñāna}).

Then, in the external enclosure, in the directions of south-east and so forth, [the \emph{yogin}] should visualise Pukkasī, Śabarī, Caṇḍālī, and Ḍombī: they arise from \emph{ū}, \emph{ṛ}, \emph{ṝ}, and \emph{ḷ}; they are sealed by Akṣobhya, Vairocana, Ratnasambhava, and Amitābha; and they are of the nature of earth, water, fire, and wind.

Next, at the doors of the east and so forth, Gaurī, Caurī, Vettālī, and Ghasmarī should be meditated on: they arise from the vowels \emph{ḹ}, \emph{e}, \emph{ai}, and \emph{o}; they are sealed by the seals of Pukkasī and so forth; they are of the nature of visual form, sound, scent, and taste.
After this, below and above, [the \emph{yogin}] should meditate on Bhūcarī and Khecarī: they are sealed by delusion (Vairocana) and passion (Amitābha); they arise from \emph{au} and \emph{aṃ}; they are of the nature of touch and the mental sphere; and they are of the essence of existence and \emph{nirvāṇa}.

And here, immediately after the goddesses arise, as they receive the consecrations of their given families, [the \emph{yogin}] should be aware of the seals of their given family. 

These fifteen \emph{yoginī}s are sixteen years of age.
They each stand on a sun disc and have the colour of mixed collyrium (\emph{bhinnāṅjana}).
They are of the nature of \emph{bodhicitta}.
Their upward-streaming hair is flaming (\emph{jvalita}) and tawny (\emph{piṅgala}).
They each bear a skull (\emph{kapāla}) measuring four finger breadths (\emph{aṅgula}) \crux on their pallets (\emph{tāluka})\crux\ which has the nature of Vajrasattva, and they each support five dry (\emph{śuṣka}) skulls (\emph{muṇḍa}) on their heads (\emph{śiras}) which are of the nature of the five buddhas.
They each have three round, red eyes.
Their faces bare fangs (\emph{daṃṣṭrākarāla}).
They are adorned with garlands of fifteen dry skulls that have the fifteen mother goddesses (\emph{mārṭrikā}) as their nature.
Their hips (\emph{kaṭi}) and buttocks (\emph{nitamba}) are covered (\emph{āvṛta}) with a tiger skin (\emph{vyāghracarman}).
Standing in the dancing posture of one leg being drawn in and the other extended (\emph{ardhaparyaṅkanāṭya}), they are mounted atop corpses and bear the five \emph{mudrā}s. 
Of these,

\begin{quote}
	Akṣobhya is in the form of the chaplet, Amitābha as earrings.
	Ratnasambhava is in the necklace, and Vairocana is present on the forearms [as a bracelet]. (1)
	
	Amoghasiddhi is in the belt, and likewise the Vajradhara is in all limbs.
	The chaplet is on the head to pay obeisance to the teacher, preceptor, and personal deity. (2)

	The earrings are to be worn on the ears so as not to hear the guru or Vajradhara spoken ill of. and the necklace is to recite mantra. (3)

	The belt is to partake in a \emph{mudrā}, the necklace to abandon the taking of life. 
	[The \emph{yoginī}s] wear anklets and bracelets are worn, and their black colour is out of friendliness. (4)
\end{quote}

Their red-tawny colour makes known their having great passion, and in order to demonstrate their wrath, [their hair] is pointing upwards.
Because their body, speech, and mind has great passion as its nature, their eyes naturally have great passion.
Their two arms are of the nature of existence and \emph{nirvāṇa}.
They have choppers in order to cut the faults of pride and the like.
They each have a skull that corresponds to the pure nature of the three realms: it is filled with the blood of the four \emph{māra}s—the aggregates and so forth; it is has three portions; and it is the embodiment of all discursive thought.
Their bending at three places (\emph{tribhaṅga}) corresponds to the pure nature of the \emph{dhrarma}-, \emph{sambhoga}-, and \emph{nirmāṇakāya}s.
The slender body corresponds to the pure nature of the \emph{svābhāvikakāya}.
They have a tiger-skin garment in order to make known their being free of veils.
Their legs are \crux in the \emph{ardhaparyaṅka} posture \crux is to make known the objectlessness of the three worlds.
In order to make known single-flavouredness, they press on the ground with one food.

After that, using light rays in five forms that emerge the seed syllable at his heart, [the \emph{yogin}] should draw in the Nairātmyā circle which has as its nature the wisdom deity (\emph{jñānasattva}) residing in Akaniṣṭha heaven and make it enter his heart's seed syllable.
After making the wisdom deity and pledge deity one, he will, in giving rise to the pride of Nairātmyā, become equal to Nairātmyā.

And here, for the sake of establishing the six-branched yoga, [the \emph{yogin}] should successively meditate on the colours black, red, yellow, green, blue, and white.

Now, as one proceeds towards excellence in [this] meditation, first it appears like the moon covered by clouds.
When it is even more excellent than that, it appears like an illusion.
When even more excellent than that, it shines forth like a dream.
Subsequently, when excellence [in meditation] is fully matured, one succeeds as a \emph{yogin} of \emph{mahāmudrā}, achieving the non-difference of the sleeping and waking states.
This is the stage of arising.

Elsewhere, the profound stage of arising [is taught] as follows: The \emph{yogin} should see in an instant precisely the \emph{bodhicitta} that, through the union of \emph{bola} and \emph{kakkola}, has the nature of Great Bliss is in between Supreme [Bliss] and [the Bliss] of Cessation, which is comprised of fifteen parts/\emph{kalā}s, as taking the form of the fifteen \emph{yoginī}s, who have the above-taught colour, insignia, and form, for that [\emph{bodhicitta}] too has as its nature the five aggregates, four elements, six objects, and body, speech, and mind.

The stage of the arisen [is taught] as follows: in an instant [the \emph{yogin}] should see the \emph{maṇḍala} circle comprised of the fifteen \emph{yoginī}s simply while \crux not \crux focusing on a seed syllable.

Likewise, there is the stage of completion (\emph{pariniṣpannakrama}):
As is well known, when the vajra body is empowered by awareness, the thirty-two chanels flow from the place of Great Bliss, and they are the thirty-two \emph{yoginī}s; thus, the body itself is comprised of Nairātmyā's circle.
To explain: \emph{lalanā} and \emph{rasanā}, residing on the left and right flanks from the below the nose down to the navel, are called the sun and the moon.
Below the navel, precisely these two, the \emph{yoni} chanels, are [called] \emph{lalanā} and \emph{rasanā}, conveying \emph{akṣobhya} (urine) and blood.
The \emph{avadhūtī}, residing in the middle of the head, throat, heart, navel, and genitals, carries \emph{bodhicitta}.
These chanels are Nairātmyā.

The \emph{abhedyā} and \emph{sūkṣmā} channels, which reside at the top of the head and nourish, number for number, nails, teeth, head hair, body hair, are Vajrā.
The \emph{divyā} channel, which is in the right ear and nourishes body hair/skin, and the \emph{vāmā} channel, which is in the back bone and nourishes flesh, are Gaurī.
The \emph{vāmanī} and \emph{kūrmajā} channels, located in the left ear and middle of the brows and nourishing tendons and the skeleton, is Vārī.
The \emph{bhāvakī} and \emph{sekā} channels, which reside in the eyes and the under arms and nourish the kidneys and heart, are Ḍākinī

The \emph{doṣāvatī} and \emph{mahāviṣṭā} channels, which are in the armpit and breasts and nourish the eyes and bile, are Pukkasī.
the \emph{mātarā} and \emph{sarvarī} channels, which reside in the navel and the tip of the nose and nourish lungs and intestines network, are Śabarī.
the \emph{śītadā} and \emph{uṣmā} channels, which reside in the mouth and in the throat and nourish the sinews of the rib area and the stomach, are Caṇḍālī.
The \emph{pravaṇā} channel in the heart, which nourishes excrement, and the \emph{hṛṣṭavadanā} channel in the genitals, which moves towards the part of the hair, are Dombī.

The \emph{svarūpiṇī} and \emph{sāmānyā} channels in the penis and the anus, which nourish phlegm and bile, are Gaurī.
the \emph{hetudāyikā} and \emph{viyogā} channels in the thighs and the shanks, which nourish blood and sweat, are Caurī.
The \emph{premaṇī} and \emph{siddhā} channels in the big toes and the back of the feet, which nourish fat and tears of sorrow, are Vettālī.
The \emph{pāvakī} and \emph{sumanā} channels are in the thumbs and the knees.
Of these, the former nourishes saliva and the latter nourishes mucus of the nose.
These two are Ghasmarī.

At the heart lotus's central portion (\emph{karṇika}) and its petals to the east and so are [the \emph{channels}] \emph{trivṛttā}, \emph{kāminī}. \emph{gehā}, and \emph{caṇḍikā}.
Of these, the first pair of channels is Bhūcarī, and the remaining ones are Khecarī.
And here, a channel that produces, nourishes, or leads to something is [said to be] its conveyer (\emph{tadvāhā}).

Furthermore, having the nature of body, speech, and mind, the dharma-, \emph{sambhoga-}, and \emph{nirmāṇa}[\emph{kāya}]s, and the three existences, these [channels] are individually established in the four \emph{cakra}s in bodies.
Of these [\emph{cakra}]s, the \emph{cakra} of production has sixty-four variagated petals, has a black \emph{a} as its seed syllables, and is established facing upwards below the navel.
The \emph{cakra} of dharma is a white eight-petaled lotus, has a black \emph{hūṁ} as its seed syllable, and is established in the heart region.
In the throad is the \emph{cakra} of enjoyment, which has sixteen red petals and a red \emph{oṁ} as its seed syllable.
At the head is the \emph{cakra} of Great Bliss: it has twelve white petals and a white \emph{haṃ} as its seed syllable, and it faces dowardwords with a stream of nectar flowing from it.
Here the individual establishment of the Blisses, Moments, and so forth should be known based on a teacher's key instructions.

The direct realization of bodhicitta that brings one to the non-duality of effortless coalesence is the innate stage. 

Next, one who is tired from meditation should repeat mantras, giving rise fo the pride of Nairātmyā.
For this, the following is innately established mantra beginning with \emph{a} and having nature of the fifteen vowels, which has \emph{oṁ} at its start and \emph{svāhā} at its end—namely, \emph{oṁ a ā i ī u ū ṛ ṝ ḷ ḹ e ai o au aṃ kvāhā}.

Giving rise to the pride of the red Nairātmyā, one should recite the \emph{purakṣobha} mantra.
For that the mantra is as follows: \emph{oṁ a ka ca ṭa ta pa ya śa svāhā}.

The mantra for the empowerment of pledge [substances] is \emph{oṁ āḥ hūṁ}.

After that, in the state of the \emph{maṇḍala} circle being manifest, [the \emph{yogin}] should remove the breats and form a \emph{bola} in the middle of the \emph{kakkola} of his body.
He should make the two shores (\emph{pārśvadvaya}) into the bell.

\begin{quote}
	Having dilligently understood the essence of the teachings (\emph{śāstra}) in the Mantra Vehicle from my guru, I have compassionately composed this illuminator of Nairātmyā.

	Alas, how can beings here have certainty about this profound \emph{maṇḍala} circle if they don't have their heads dirited by the dust of lord Śabara's lotus feet?

	By the merit obtained in this spreading out of the \emph{abhisamaya}, may all living beings stand firm (lit.\ have as their nature the standing firm) in the state of Vajradhara, having eliminated the pain and suffering of the three worlds. 
\end{quote}


\end{document}







