\documentclass[naipra.tex]{subfiles}
\begin{document}
\beginnumbering
\begin{sanskrit}
\pstart\noindent
\fnum{\MSN}{260}{r5}oṁ namaḥ śrīnairātmyāyai |\footnote{\begin{english}
	The opening \emph{oṁ namḥ śrīnairātmyāyai} is a scribal homage.
\end{english}}
\pend

\medskip\versequote
parihṛtaparikalpaṃ dharmakāyaṃ yam āhur &
\hspace{20pt}nirupamasukhamātraṃ cāru sambhogakāyam | \&
\versequote
bhuvanahitavidhānād yasya nirmāṇakāyaṃ &
\hspace{20pt}bhavatu sa bhagavān vaḥ śreyase vajrasattvaḥ ||\footnote{
	\begin{english}%
	This verse, in Mālinī metre, serves as a \emph{maṇṅgalācaraṇa} in other texts attributed to Advayavajra.
	We find it in Mar pa chos kyi blo gros's Tibetan translation of the \emph{Saptākṣarasādhana}: \emph{kun du rtog pa yongs su spangs pa'i chos skur gang gsungs dang/ /dpe med bde ba rtsal gyis mdzes pa longs spyod rdzogs sku dang/ /gang gi gnas la phan par mdzad pa las ni sprul pa'i sku/ /bcom ldan rdo rje sems dpa' de yis khyed la bde legs shog /} (D f.\ 130r–v) (read \emph{rtsal} as \emph{tsam}).
	It is absent from the text as printed in Bhattacaryya's edition of the \emph{Sādhanamālā}.

	The verse is also found at the beginning of Advayavajra's \emph{*Śrīcakrasaṃvaropadeśa}, for which rMa ban chos 'bar's translation reads: \emph{gang zhig kun du brtags pa yongs su spangs pa'i chos sku dang/ /dpe med bde ba tsam gyis mdzes pa'i longs spyod rdzogs sku dang/ gang gi thugs rje sa rnams phan mdzad sprul pa'i sku brjod pa/ /bcom ldan rdo rje sems dpa' de yis khyod la bde legs shog /} (D f.\ 139r).

	The verse is also transmitted in the so-called \emph{Sādhanavidhāna} codex, on fol.\ 3r, in an \emph{adhyātmahomavidhi}.
	Péter-Dániel Szántó (personal communication) surmises that the colophon to this brief text is written in old Newar and amounts to saying that the \emph{vidhi} was extracted from a \emph{ṭippanī} on the \emph{Saṃvarodayatantra}. 
	The only noteworthy variant reading here is `\emph{sa bhavatu bhagavān}' in place of `\emph{bhavatu sa bhagavān}', both of which are equally plausible.

	It may be worth comparing the two above Tibetan translations of this verse with Ye shes 'byung gnas's effort here for the NaiPra: \emph{kun du rtog pa yongs spangs chos skur gang brjod pa/ /dpe med bde ba tsam mdzes longs spyod rdzogs pa'i sku/ /sa yi phan pa'i rgyur gyur gang gi sprul pa'i sku/ /bcom ldan rdo rje sems dpa' khyod des dge bar shog /}. 
	The translation \emph{rgyur gyur} for \emph{vidhānāt} is difficult to account for; it also appears to be an adjective qualifying either \emph{gang} or \emph{sprul pa'i sku} rather than an ablative form (here it is perhaps Marpa's translation that has the clearest rendering with `\emph{mdzas pa las}').
	Similarly, the syntax of the final line, with \emph{rdo rje sems dpa'} separated from the pronoun \emph{de}, is considerably more opaque than the other two translations.

	Although it may be entirely coincidental, it is nonetheless noteworthy that the opening verse of Ratnākaraśānti's \emph{Bhramaharasādhana} is also in Mālinī metre.
\end{english}} \&

% vidhānāt = according to the principle of what is beneficial for [various] world realms. 

% kun du rtog pa yongs spangs chos skur gang brjod pa/ /  
% dpe med bde ba tsam mdzes longs spyod rdzogs pa'i sku/ /  
% sa yi phan pa'i rgyur gyur gang gi sprul pa'i sku/ /  
% bcom ldan rdo rje sems dpa' khyod des dge bar shog /  

\medskip\versequote
ekatra vitataṃ \vedtext{spaṣṭam abodha}{
	\lemma{spaṣṭam abodha\abbr}	
	\variants{\emd ; spaṣṭaṃm abodha\abbr\ \MSN}
}laghuvistaram |\footnote{\begin{english}
	\TIB\ renders the first two \emph{pāda}s as follows: \emph{gcig tu gsal la rgyas pa yis/ /mi rtog nyung la rgyas pa dag/ /}.
	If the translation has been transmitted correctly here, I am uncertain what meaning this was intended to convey. 
	I understand, somewhat tentatively, the Sanskrit text as edited here in the following sense: `[A \emph{sādhana} which is] clearly (\emph{spaṣṭa}—to be taken as an adverb, adjective, or both) laid out in a single place, with a small amount of prolixity for those who lack understanding.'
\end{english}} & 
nairātmyāsādhanaṃ brūmo yathāmati yathāgamam || \&

% gcig tu gsal la rgyas pa yis/ /  
% mi rtog nyung la rgyas pa dag /  
% ji ltar blo bzhin lung ji bzhin/ /  
% bdag med ma yi sgrub thabs bshad/ /  

\medskip\pstart
yogī khalu \vedtext{śmaśānādau}{
	\variants{\emd ; śmaśānādi \MSN}
} manonukūle sthāne pañcāmṛtādisamayasevī \vedtext{sukhāsanopavi\fnum{\MSN}{260}{v}ṣṭo}{
	\lemma{sukhāsanopaviṣṭo}
	\variants{\emd ; sukhāsanopaviṣṭaṣṭo \MSN}
} niḥsaṅgo \vedtext{niḥśaṅkaḥ}{
	\variants{\emd ; niḥśaṅka \MSN}
} \vedtext{sattvārthodyatamatir}{
	\variants{\emd ; satvā((r))thodyamati \MSN}
} nairātmyāhaṃkāram utpādya, hṛtsūrye nīla\dsh \begin{mantra}hūṁ\end{mantra}\dsh kāraṃ dhyāyāt. 
\vedtext{tatas taddīptai}{
	\variants{\conj\ (\TIB\ de nas de'i gsal ba'i); tatasthadīpai \MSN}
} raśmibhis traidhātukam avabhāsamānair ākṛṣya, akaniṣṭhabhuvanavartinam\footnote{\begin{english}
	\TIB\ reflects a plural form of \emph{akaniṣṭhabhuvanavartin}: \emph{'og min gyi gnas na bzhugs pa rnams}.
	Given that what follows is a description of only the eight-faced Heruka, the plural form can be regarded as an error.

	\TIB\ also places \emph{'og min gyi gnas na bzhugs pa rnams} before \emph{ākṛṣya} (\emph{bkug ste}), which may be simply for syntactical naturalness in Tibetan rather than a reflection of a different reading in Sanskrit.
\end{english}} aṣṭayoginīparivṛtaṃ ṣoḍaśabhujam aṣṭāsyaṃ kapālamālāviracitaśekharaṃ catuścaraṇasamākrāntacaturmāraṃ nīlavarṇaṃ dakṣiṇakaranikarakalitakapālasakalanilīnagaja\dsh turaga\dsh kha-ra \dsh vṛṣabha\dsh karabha\dsh manuja\dsh śarabha\dsh vṛṣadaṃśam\footnote{\begin{english}
	\TIB\ lacks a clear reflex of \emph{sakalanilīna} in \emph{dakṣiṇakaranikarakalitakapālasakalanilīna°}: \emph{g.yas pa'i phyag gi tshogs rnams kyis bsnams pa'i thod pa rnams su gnas pa}.
	The translation \emph{gnas pa}, for which one might expect the Sanskrit \emph{sthita}, is nonetheless perhaps a loose rendering of \emph{sakalanilīna}.
\end{english}} itarapāṇikadambagatapadmabhājanavartidharaṇi\dsh varuṇa\dsh samīraṇa\dsh jvalana\dsh rajanīnātha\dsh taraṇi\dsh yama\dsh dhana-\\ daṃ kṛṣṇapradhānavadanam indukundāvadātadakṣiṇamukham\footnote{\begin{english}
	\TIB\ lacks a reflex of \emph{avadāta} within the compound \emph{indukundāvadātadakṣiṇamukha}: \emph{g.yas pa'i zhal ni zla ba dang/ kun da lta bu'o/ /}
\end{english}} atimātralohitavāmavadanam atidhūmravikarālordhvavaktram \vedtext{atimalinetara}{
	\lemma{atimalinetara\abbr}
	\variants{\conj\ (\TIB\ shin tu gnag pa); alamalinetara\abbr\ \MSN}
}sakalavadanaṃ śatārdhamuṇḍamālālaṅkṛtaṃ nairātmyā\vedtext{liṅgitakandharam}{
	\lemma{\abbr liṅgitakandharam}
	\variants{\emd ; °liṃgitaṃkandharam \MSN}
} ambarataralavartinam agrato dhyāyāt. 
\pend


% rnal 'byor pas yang dur khrod la sogs pa yid du 'ong ba'i gnas su bdud rtsi la sogs pa'i dam tshig bsten [D f. 219r]*/ /nas bde ba'i stan la nye bar 'dug ste/ zhen pa med cing dogs pa dang bral bas sems can gyi don du blo bskyed de bdag med ma'i nga rgyal bskyed par bya'o/ /snying gar nyi ma'i dkyil 'khor la yi ge h'um sngon po bsgoms te/ 

% de nas de'i gsal ba'i 'od zer rnams kyis khams gsum snang bar byas nas 'og min gyi gnas na bzhugs pa rnams bkug ste/ rnal 'byor ma brgyad kyis yongs su bskor ba phyag bcu drug pa zhal brgyad pa/ thod pa'i phreng bas rnam par spras pa'i thod can zhabs bzhi yis bdud bzhi mnyam par mnan pa kha dog sngon po/ g-yas pa'i phyag gi tshogs rnams kyis bsnams pa'i thod pa rnams su gnas pa ni glang po che dang/ rta dang/ bong bu dang/ glang dang/ rnga mo dang/ skyes pa dang/ seng ge dang/ byi la rnams so/ /cig shos kyi phyag gi tshogs rnams su son pa'i padma'i snod rnams su gnas pa ni/ /'dzin pa dang/ chu dang/ skul byed dang/ 'bar ba dang/ mtshan mo'i mgo dang/ nyi ma dang/ gshin rje dang/ gnod sbyin rnams so/ /zhal gyi gtso bo ni nag po'o/ /g-yas pa'i zhal ni zla ba dang/ kun da lta bu'o/ /g-yon pa'i zhal shin tu dmar ba'o/ /steng gi zhal dub lta bu shin tu gtsigs pa'o/ /zhal gzhan mtha' dag ni shin tu gnag pa'o/ /brgya phyed kyi thod pa'i phreng bas brgyan pa'o/ /bdag med mas mgul nas 'khyud pa/ nam mkha'i dkyil na bzhugs pa mdun du bsgom par bya'o/ /

\pstart
tadanantaraṃ \vedtext{bāhyaguhyatattvapūjābhir}{
	\variants{\emd\ (\TIB\ phyi dang/ gsang ba dang/ de kho na nyid rnams kyis mchod pa); bāhyapūjāguhyatatvapūjābhir \MSN}
} aṣṭayoginībhiḥ pūjayet.
atra ca prajñopāyayos \vedtext{tādā[tmyenāvabodhanāya picuvajrasya pūjanam. 
tato vandanaṃ pāpadeśanāpāpākaraṇasaṃvaraṃ puṇyānumodanāpuṇyapariṇāmanātriśaraṇagamanabodhicittotpādā\hspace{0em}]\hspace{0em}tmabhāva}{
	\lemma{tādātmyenāvabodhanāya \ldots\ °cittotpādātmabhāva°}
	\variants{\diag\ (\TIB\ [\emph{starting from} atra ca \emph{and ending} ca kṛtvā]: 'di la yang thabs dang shes rab dag gis de'i bdag nyid rtogs par bya ba'i phyir pi tsu badzra'i mchod pa'o/ /de nas phyag 'tshal ba dang/ sdig pa bshags pa dang/ sdig pa slan chad sdom pa dang/ bsod nams la rjes su yi rang ba dang/ bsod nams yongs su bsngo ba dang/ gsum la skyabs su 'gro ba dang/ byang chub tu sems bskyed pa dang/ bdag nyid kyi dngos po dbul ba dang/ gsol ba gdab pa byas te/); tādātmabhāva° \MSN}
}niryātanādhyeṣyaṇāś\footnote{\begin{english}
	Here the transmitted text has suffered from what was likely a scribe's eyeskip.
	I offer this conjectural reading based on \TIB\ as well as similar formulations in two other \emph{sādhana}s by Advayavajra.
	First, the \emph{Saptākṣarasādhana} reads: \emph{\ldots\ yathāvidhinā pūjayet vandayet | tatas teṣāṃ purataḥ pāpadeśanāpāpākaraṇasaṃvaraṃ puṇyānumodanātriśaraṇagamanabodhicittotpāda\dsh ātmabhāvaniryyātanā\dsh adhyeṣaṇāyācanāś ca kṛtvā \ldots } (ed.\ p.\ 460).
	In the NaiPra, the Tibetan translation indicates the inclusion of \emph{puṇyapariṇāmanā}, but the word \emph{yācanā} is reflected neither in the Tibetan nor in the Sanskrit manuscript, which has resumed at the place where one would expect to see it.
	Otherwise the two texts are evidently closely parallel.
	Advayavajra's \emph{Vajravārāhīsādhana} reads as follows: \emph{\begin{sanskrit}tadagrataḥ pāpadeśanāpāpākaraṇasaṃvarapuṇyānumodanāpuṇyapariṇāmanātriśaraṇagamanabodhicittotpādādikaṃ kṛtvā\end{sanskrit}} \ldots\ (ed-f p.\ 59; ed-b p.\ 424) (\emph{°pāpākaraṇasaṃvara°}] ed-f; \emph{deest} in ed-b but recorded as a variant).
	Here \emph{vandana} is not present, and \emph{ātmaniryātana} and so forth have probably been replaced with the word \emph{ādika}.
	In the NaiPra, the \emph{ca} before \emph{kṛtvā} indicates that more than one word precedes, as in \emph{Saptākṣarasādhana}.
	I assume \emph{pāpadeśanāpāpākaraṇasaṃvaraṃ} can be understood as a \emph{samāhāradvandva}, but can find no other attestation of the compound, and it is not immediately clear to me why these actions should be divided in the way they are.
	On the various preliminary stages in similar \emph{sādhana}s, see \textcite[122–124]{english2002}.

	As for the first sentence (\emph{atra ca prajñopāyayoḥ} etc.), I can find no parallel in others \emph{sādhana}s, so I only rely on \TIB\ for the proposed conjecture. 
	If I have understood the Tibetan correctly, I believe Advayavajra is offering a justification for worshipping the eight-faced Heruka, i.e.\ Picuvajra, at the beginning of the \emph{sādhana}: to put it somewhat baldly, a \emph{sādhana} that includes worship (and meditation on) both male and female deities serves to help one realise insight and means as having an identical nature.
\end{english}} ca kṛtvā, caturbrahmavihārān bhāvayitvā, sakalavastutattvasāra\vedtext{saṅgrāhakātmakaṃ}{
	\lemma{°saṅgrāhakātmakaṃ}
	\variants{\emd; °saṅgrahākātmakaṃ \MSN}
} \begin{mantra}oṁ śūnyatājñānavajrasvabhāvātmako\footnote{
	\begin{english}%
		On the compound \emph{śūnyatājñānavajrasvabhāvātmaka}, traditional authorities have interpreted \emph{vajra} either as co-referential with \emph{jñāna} or with \emph{svabhāva}.
		The former interpretation is offered by, for example, Śākyarakṣita in his \emph{Abhisamayamañjarī}, and the latter by Abhayākaragupta in ch.\ 4 of his Abhayapaddhati (\cite[239–40 n.\ 273, n.\ 277]{english2002}; \cite[292]{isaacson2007}; \cite[140, 234]{yang2014}).
	\end{english}
} 'haṃ śūnyatājñānavajrasvabhāvātmakāḥ sarvadharmāḥ\end{mantra}\footnote{\begin{english}
	\TIB\ renders \emph{śūnyatājñānavajrasvabhāvātmakāḥ sarvadharmāḥ} in Tibetan, indicating that, perhaps, these words were not understood as part of the mantra: \emph{dngos po mtha' dag gi de kho na nyid kyi dngos po bsdus pa/ oṁ shū nya tā dznyā na badzra sva bhā ba ātma ko 'ham/ stong pa nyid kyi ye shes rdo rje rang bzhin gyi bdag nyid la chos thams cad ces bya ba'i sngags kyi don bsgoms pas rab tu mi gnas pa'i ngo bor gnas so/ /}
	Indeed the mantra \emph{oṁ śūnyatājñānavajrasvabhāvātmako 'ham} does generally stand on its own; however, it would appear that Advayavajra is fond of this reformulation of the mantra, which emphasizes the emptiness of all phenomena (see \cite[128]{english2002}).

	Evidence to support this can be found in other \emph{sādhana}s composed by Advayavajra, such as the \emph{Saptākṣarasādhana}: \emph{tataḥ oṁ śūnyatājñānavajasvabhāvātmakāḥ sarvadhāḥ oṁ śūnyatājñānavajasvabhāvātmako 'ham iti sakalavastutattvasārasaṃgrāhakaṃ mantrārtham āmukhīkurvvan \ldots } (ed.\ p.\ 460).
	Note, however, that Mar pa Chos kyi dbang phyug's Tibetan translation of the \emph{Saptākṣarasādhana} does not reflect the first \emph{oṁ} and appears to have attempted to interpret the words as a stand-alone clause: \emph{de nas chos thams cad ni stong pa nyid kyi ye shes kyi rdo rje'i bdag nyid de/ oṁ shū nya tā dznyā na badzra sva bhā ba ātma ko 'ham/ zhes bya ba dngos po ma lus pa'i de kho na nyid sdud par byed pa'i sngags kyi de kho na mngon du byed cing/} (D f.\ 131r) (\emph{de kho na} may be a corruption of \emph{don} or \emph{don kho na}).

	Advayavajra's \emph{Hevajraviśuddhinidhi} also has a formulation resembling the \emph{Saptākṣarasādhana}: \emph{etadantaraṃ \textnormal{(?)} sarvadharmapravicayalakṣaṇayā prajñayā sarvadharmān pratītyasamutpādakān svabhāvānutpannān adhimuñcan, tadarthaṃ dyotakatvāt sakalavastutattvasārasaṅgrāhakatvena ca, \begin{mantra}\textnormal{oṁ śūnyatājñānavajrasvabhāvātmakāḥ sarvadharmmāḥ | oṁ śūnyatājñānavajrasvabhāvātmako 'ham}\end{mantra} iti mantram imaṃ manasā paṭhitvā} \ldots\ (ms f.\ 66r7–v2).
	Here 'Gos lo tsā ba's Tibetan translation completely lacks any reflex of the words in question (see D f.\ 176r).

	We also find a rendering of more or less the same formulation in the Tibetan translation of Advayavajra's *\emph{Hevajrasādhana}: \emph{de ltar bla na med pa'i chos thams cad rab tu 'byed pa'i mtshan nyid kyi shes rab kyis chos thams cad rten 'brel las skyes pa tsam rang bzhin gyis gzod ma nas skye ba med par gsal bar shes par bya ste/ dngos po ma lus pa'i bde ba de kho na nyid kyi snying por bsdus pa'i stong pa'i ye shes kyi rdo rje'i rang bzhin gyi chos shes nas oṁ shū nya tā dznyā na badzra sva bhā va ā tma ko 'ham zhes pa'i sngags de yid kyis bzlas te/} (D f.\ 163r). 
	Leaving aside other slight differences for the moment, we can see that the translator (whose identity is unknown to me) appears to have treated the words in question with an approach similar to that of Mar pa Chos kyi dbang phyug.

	Finally, the extended mantra is also found in some recensions of the \emph{Vajravārāhīsādhana}. Finot's edition reads: \emph{tataḥ, oṃ śūnyatājñānasvabhāvātmakāḥ sarvvadharmmāḥ, oṃ śūnyatājñānavajrasvabhāvātmako 'ham iti mantrārtham āmukhīkurvan muhūrttam apratiṣṭharūpena tiṣṭhet} (perhaps \emph{vajra} is missing from the first \emph{śūnyatājñāna°}). Bhattacharyya's edition in the \emph{Sādhanamālā} omits the first half of the mantra, as does the translation by mTshur ston dBang gi rdo rje. Yar lung lo tsā ba Grags pa rgyal mtshan's translation of the same text, however, does reflect something of the mantra: \emph{de nas chos thams cad rdo rje'i rang bzhin gyi stong pa nyid du byas nas/ oṁ shū nya tā dznyā na badzra svabhā ba ātma ko&amp;ham/ zhes pa'i sngags kyi don mngon du byas nas skad cig gis mi gnas pa'i skur bsam par bya'o/ /}

	Taken altogether, this evidence affirms that Advayavajra had a special preference for an `enhanced' formulation of the popular mantra, which may have caused some confusion for Tibetan translators. 
	In view of this preference, it may also be worth noting that the formulation, as well as other key terms associated with Advayavajra's philosophy such as \emph{apratiśṭhita}, is wholly absent from the relevant portion of the \emph{Hevajrākhya}, which is a factor to consider when evaluating the text's authorial attribution: \emph{tataḥ paṭhed jinamantrakam—\begin{mantra}\textnormal{oṁ śūnyatājñānavajrasvabhāvātmako 'ham}\end{mantra}. tasmin samaye svaparaśaradamalanabhasannibhaṃ paśyet} (f.\ 9r). 

	One lingering doubt is that the four other attestations of the mantra appear to put \emph{sarvadharmāḥ} first and \emph{aham} second, whereas the NaiPra does the opposite.
	Given that \TIB\ and the Sanskrit witness both support this seemingly reversed order in the NaiPra, I do not emend the text; however, there is some doubt as to whether it is correct.
\end{english}} \fnum{\MSN}{261}{r}iti mantrārthaṃ bhāvayann apratiṣṭhitarūpeṇa tiṣṭhet. 
\pend

% * Ms. seems to be missing some text as found in TIb.: pi tsu badzra'i mchod pa'o/ /de nas phyag 'tshal ba dang/ sdig pa bshags pa dang/ sdig pa slan chad sdom pa dang/ bsod nams la rjes su yi rang ba dang/ bsod nams yongs su bsngo ba dang/ gsum la skyabs su 'gro ba dang/ byang chub tu sems bskyed pa dang/ 
% * sakalavastutatvasāra] ms (attested in sādhanamālā); Tib. dngos po mtha' dag gi de kho na nyid kyi dngos po 

% de'i rjes su phyi dang/ gsang ba dang/ de kho na nyid rnams kyis mchod pa lha mo brgyad po rnams kyis mchod par bya'o/ /'di la yang thabs dang shes rab dag gis de'i bdag nyid rtogs par bya ba'i phyir pi tsu badzra'i mchod pa'o/ /de nas phyag 'tshal ba dang/ sdig pa bshags pa dang/ sdig pa slan chad sdom pa dang/ bsod nams la rjes su yi rang ba dang/ bsod nams yongs su bsngo ba dang/ gsum la skyabs su 'gro ba dang/ byang chub tu sems bskyed pa dang/ bdag nyid kyi dngos po dbul ba dang/ gsol ba gdab pa byas te/ tshangs pa'i gnas bzhi bsgom par bya'o/ /dngos po mtha' dag gi de kho na nyid kyi dngos po bsdus pa/ aom sh'u n+ya t'a dzny'a na badzra sva bh'a ba [D f. 219v] a'atma ko// ham/ stong pa nyid kyi ye shes rdo rje rang bzhin gyi bdag nyid la chos thams cad ces bya ba'i sngags kyi don bsgoms pas rab tu mi gnas pa'i ngo bor gnas so/ /

\pstart
tataḥ praṇidhim anusmṛtya, samādher vyutthāya, repheṇa purataḥ sūryamaṇḍalaṃ dhyātvā, \vedtext{tatra}{
	\lemma{tatra}
	\variants{\emd ; tatra ((tya/tpa)) \MSN}
} \begin{mantra}hūṁ\end{mantra}\dsh kāreṇa viśvavajraṃ ca dhyātvā, tato viśvavajrāt \vedtext{sphuradbhir aṃśusaṃhatair vajrair}{
	\variants{\conj ; sphuradbhir aṇusaṃhater vvajraiḥ \MSN ; \emph{cf.\ }\TIB : 'od zer gyi rdul phra rab kyi tshogs 'phros pa des}
}\footnote{\begin{english}
	One may consider correcting the manuscript's reading \emph{aṇusaṃhater vvajraiḥ} to \emph{aṇusaṃhatair vajraiḥ}, but I do not believe this yields adaquate sense.
	The Tibetan translation reads \emph{'od zer gyi rdul phra rab kyi tshogs 'phros pa des}, and on this basis we may conjecture a reading along the lines of \emph{sphuradbhir aṇusaṃhatair raśmibhiḥ} (`with light densely packed with particles'). 
	This not a bad interpretation, but I do wonder if it was possibly the Tibetan translator's own conjecture, made in an attempt to understand \emph{aṇusaṃhataiḥ}.
	The reading I suggest here is partially inspired by Ratnākaraśānti's MuĀ, commenting on HeTa 1.3.3:

	\begin{quote}
		\emph{purastād agnivarṇena repheṇa sūryamaṇḍalaṃ dhyātvā tanmadhye hūṃkāreṇa viśvavajraṃ vicintya tatkiraṇasūkṣmavajraiḥ sphuradbhiś caturdiggatair atyantaṃ ghanībhāvāt prākāraṃ bhāvayet} (as cited in \cite[293]{isaacson2007}).
	\end{quote}
	
	\noindent We see that it is indeed additional vajras from the initial vajra that produce the fence, and these are described as `[made of] the light rays of that [initial vajra]'.
	Since the characters for \emph{ṇu} and \emph{śu} can be extremely similar, I regard this conjecture as relatively minor intervention.
\end{english}} \vedtext{vajraprākāraṃ}{
	\variants{\emd ; vajraprākārai \MSN}
} pañjarabandhanam\footnote{
	The compound \emph{pañjarabandhana} can be understood as a \emph{dvandva}.
	This section of the \emph{sādhana} derrives from HeTa 1.3.3, where the form \emph{pañjarabandhanaṃ} (or \emph{pañjara bandhanaṃ}) appears to be \emph{metri causa}: \emph{repheṇa sūryaṃ purato vibhāvya tasmin ravau hūṁbhavaviśvavajram | tenaiva vajreṇa vibhāvayec ca prākārakaṃ pañjarabandhanaṃ ca ||}
} adho vajramayīṃ \vedtext{bhūmiṃ}{
	\variants{\emd ; bhūmi \MSN}
} parikhāṃ ca \vedtext{vicintayet}{
	\variants{\emd ; viñcintayet \MSN}
}.
raviviśvavajrābhyāṃ ca raśmībhūya samantataḥ prasṛtābhyāṃ tat sarvaṃ dṛḍhīkuryāt.\footnote{
	For the final sentence of this paragraph—\emph{raviviśvavajrābhyāṃ ca raśmībhūya samantataḥ prasṛtābhyāṃ tat sarvaṃ dṛḍhīkuryāt}—\TIB\ yields a different sense, which is probably the result of either very free translation or misunderstanding: \emph{nyi ma dang sna tshogs rdo rje dag las 'od zer dpag tu med pa byung bas de ltar de dag thams cad brtan par bya'o/ /}
}
\pend

% Tib. is basically a paraphrase for the last sentence, and in that sense it can be seen as kind of correct.

% de nas smon lam gyis dran pas langs te/ ram las mdun du nyi ma'i dkyil 'khor blta'o/ /de la yi ge h'um las sna tshogs rdo rje bsam mo/ /de nas de'i sna tshogs rdo rje'i 'od zer gyi rdul phra rab kyi tshogs 'phros pa des rdo rje ra ba dang dra ba bcing ngo/ /'og tu rdo rje'i rang bzhin gyi sa gzhi dang/ 'obs bsam par bya'o/ /nyi ma dang sna tshogs rdo rje dag las 'od zer dpag tu med pa byung bas de ltar de dag thams cad brtan par bya'o/ /

\pstart
tadanantaraṃ khadhātau dharmodayākārām antaḥsuṣirām atibahaladhavalām ūrdhvāṃ\footnote{\begin{english}
	Where the Sanskrit manuscript transmits the reading \emph{ūrdhvāṃ}, \TIB\ translates \emph{steng yang pa}, perhaps reflecting \emph{upari viśālām}.
	This latter reading has the advantage of many parallels in other descriptions of the \emph{dharmodaya}, such as the one found in Advayavajra's \emph{Hevajraviśuddhinidhisādhana} (f. 67v): \emph{adhaḥ sūkṣaṃ upari viśālaṃ trikoṇam}.
	Perhaps, however, Advayavajra wrote \emph{ūrdhvām} (`upright') as a more telegraphic description of the shape; otherwise, if we read \emph{upari viśālāṃ}, one might also expect to see here the words \emph{adhaḥ śūkṣām}. 
\end{english}} prajñāṃ paśyet.
tatas tadantarvarti viśvavarṇ\vedtext{āṣṭadalaṃ}{
	\lemma{°āṣṭadalaṃ}
	\variants{\emd ; °āṣṭadala \MSN}
} \vedtext{viśālaṃ}{
	\variants{\emd ; viśāla \MSN ; \emph{not reflected in \TIB }}
} kamalaṃ dhyāyāt.
tatas tanmadhye\footnote{\begin{english}
	\TIB\ renders \emph{tanmadhye} and \emph{tadantarvarti} of the previous sentence as \emph{de'i steng du}.
	It renders \emph{madhyavarti} in the following compound as \emph{la gnas pa}.  
	The translator appears, therefore, to have made a conscious decision to avoid translating these with words meaning `inside' or `in the middle of'.
\end{english}} rephodbhavasūryamaṇḍalamadhyavarti\dsh \begin{mantra}hūṁ\end{mantra}\dsh kārapariṇataṃ\footnote{
	\TIB\ here and again below renders \emph{repha} as \emph{raṃ}.
} viśvavajraṃ cintayet. 
\vedtext{viśvavajramadhye}{
	\lemma{viśvavajramadhye}
	\variants{\MSN\PCreading ; vijaśvavajramadhye \MSN\ACreading}
} ca mārutatejo\vedtext{\hl{jalāvanīr}}{
	\lemma{°jalāvanīr}
	\variants{\emd ; °jalāvani \MSN}
}\footnote{\begin{english}
	\hl{QUESTION} Is it correct that \emph{māruta°} is a \emph{dvandva}, and that \emph{dhūmra°} etc.\ should be implicitly qualified by a word such as \emph{maṇḍalāni}? I.e., `One should visualize wind ... as [\emph{maṇḍala}s] that are grey, ....' See Tib: \emph{sna tshogs rdo rje'i dbus su yang rlung dang/ me dang/ chu dang/ sa rnams ni du ba dang/ dmar po dang/ dkar po dang/ ljang gu rnams/ gzhu dang/ zur gsum pa dang/ zlum po dang/ gru bzhi rnams/ yam ram bam lam rnams yongs su gyur pa las steng nas steng du blta'o/ /}
\end{english}} dhūmraraktaśuklaharitāni dhanustrikoṇaparimaṇḍalacaturasrākārāṇi\footnote{
	\TIB\ lacks a reflex of \emph{ākāra} at the end of this compound: \emph{\ldots\ zlum po dang/ gru bzhi rnams/}
} \begin{mantra}yaṁ\dsh raṁ\dsh vaṁ\dsh laṁ\dsh \end{mantra}pariṇatāni upary upari paśyet. 
etat sarvaṃ \vedtext{jñānamātram}{
	\lemma{jñānamātram}
	\variants{\emd ; jñānamātraṃm \MSN}
} ākalayan tatpariṇataṃ caturasraṃ caturdvāram\footnote{\begin{english}
	After \emph{caturdvāram}, \TIB\ reads \emph{rta babs bzhi pa}, reflecting \emph{catustoraṇam}.
	This word is certainly fitting, but at present I feel it is impossible to say whether it was added to the translation or lost from the Sanskrit witness.

	The text here verges on entering \emph{anuṣṭubh} metre, as it inspired by verses that can be traced back to at least the \emph{Sarvatathāgatatattvasaṅgraha}, and which are often quoted or employed with variations in countless texts (see \cite[143 n.\ 24]{tribe2016} for references, and p.\ 254–5 for Vilāsavajra's version of these in the \emph{Nāmamantrārthāvalokinī}; see also, for example, HeTa 1.10.21).
\end{english}} aṣṭastambhopaśobhitaṃ hārārdhahārabhūṣitaṃ kūṭāgāraṃ paśyet. 
\pend


% de'i rjes su nam mkha'i dbyings chos 'byung gi rnam pa nang khong stong shin tu dkar po steng yangs pa'i shes rab blta bar bya'o/ /de nas de'i steng du kha dog sna tshogs pa'i padma 'dab ma brgyad pa bsgom par bya'o/ /de nas de'i steng du ram las byung ba'i nyi ma'i dkyil 'khor la gnas pa'i yi ge h'um yongs su gyur pa las sna tshogs rdo rje bsam par bya'o/ /sna tshogs rdo rje'i dbus su yang rlung dang/ me dang/ chu dang/ sa rnams ni du ba dang/ dmar po dang/ dkar po dang/ ljang gu rnams/ gzhu dang/ zur gsum pa dang/ zlum po dang/ gru bzhi rnams/ yam ram bam lam rnams yongs su gyur pa las steng nas steng du blta'o/ /de dag thams cad ye shes tsam du btags te/ de dag yongs su gyur pa las/ gru bzhi pa/ sgo bzhi pa/ rta babs bzhi pa/ ka ba brgyad kyis nye bar mdzes pa/ dra ba dang dra ba phyed pas brgyan pa'i gzhal yas khang blta bar bya'o/ /

\pstart
tataḥ prākārābhyantare 'ṣṭa śmaśānāni cintayet.
atra\footnote{
	\TIB\ lacks a reflex of \emph{atra}.
	Perhaps \emph{tatra} with a partitive sense would read better here, but I don't see very strong reasons to discount \emph{atra}.
} pūrve devendro \vedtext{harītakīvṛkṣe}{
	\variants{\emd ; harītakīvṛkṣa \MSN}
}\footnote{\begin{english}
	For each of the trees in the eight charnel grounds, \TIB\ treats the words as if they were nominative forms.
	\MSN\ offers two instances of the words without case endings, one instance with a form that is corrupt in another way, and five instances of locative forms.
	If we were to accept nominative forms, a conjunction such as `\emph{ca}' would also be natural; I therefore find this possibility unlikely, and adopting it would require major alterations to the transmitted text. 
	Compounded forms such as \emph{harītakīvṛkṣamecakavarṇaḥ} can, I believe, also be discounted, as they appear unprecedented and unnecessary.
	If locative forms were intended, we can account for the error in that it appears relatively easy for a scribe to mistake \emph{kṣa} for \emph{kṣe}; we also see that forms less susceptible to this confusion (e.g.\ \emph{śākhini} or \emph{tarau}) are here unambiguously locative. 

	According to other accounts of the eight charnel grounds, there is a \emph{śirīṣa} in the east (see \cite[vol.\ 2 739–740]{gerloff2020}; \cite[140]{english2002}).
	The Pandanus Database of Plants identifies \emph{śirīṣa} as \emph{Acacia lebbeck Willd.} (Siris tree) and \emph{harītakī} as \emph{Terminalia chebula Retz.} (Chebulic myrobalan).
	It is unclear to me whether or not Advayavajra regarded the two as synonyms.
	The anonymous \emph{Aṣṭaśmaśāna} also identifies a \emph{harītakīvṛkṣa} in the east.
\end{english}} mecakavarṇo\footnote{
	\begin{english}%
		Here and below in the list of cloud colours \TIB\ renders \emph{mecakavarṇa} (a dark colour) as \emph{ser po} (yellow).
		The \emph{Aṣṭaśmaśāna}, for instance, specifies that the \emph{yakṣa} is white, whereas Indra is yellow.
		These colours should be understood as qualifying the \emph{yakṣa}s (also called \emph{maharddhika}s or \emph{kṣetrapāla}s in some sources) and not the \emph{dikpati}s, whose colours are described later as being `as they are commonly known'.
		The colours given by Advayavajra throughout this passage do not concord with the colours of the \emph{yakṣa}s given in the \emph{Aṣṭaśmaśāna}.
	\end{english}
} dantivadanaḥ.
dakṣiṇe \fnum{\MSN}{261}{v}yamaś \vedtext{cūtavṛkṣe}{
	\variants{\emd ; cūtavṛkṣa \MSN}
} mahiṣānanaḥ sitavarṇaḥ.
paścime 'śokatarau varuṇo raktaḥ siṃhamukhaḥ.\footnote{
	The syntax of the formulation changes with the third \emph{śmaśāna}, which is also reflected in \TIB .
	Other accounts of the eight charnel grounds, including visual depictions, make it clear that it is the \emph{yakṣa}s who are \emph{in} the trees, whereas the \emph{dikpāla}s are nearby the threes—for example, the \emph{Aṣṭaśmaśāna} reads:

	\begin{quote}
		\emph{tatrāsokavṛkṣe mahardhiko jvalākulakaraṅko nāmo makaramukho raktaḥ | vṛkṣādhasi dikpatir varuṇo nāgāsanaḥ suklaḥ |} (ms.\ 4v; ms.\ reads \emph{vāruṇo}; note the variant colours)
	\end{quote}

	\noindent Presumably the locative \emph{aṣokatarau} in this sentence should accordingly be understood in the sense of `at the Ashoka tree' rather than `in the Ashoka tree'.
}
uttarato bodhiśākhini kubero haritābho manuṣyamukhaḥ.  
\pend

\pstart
\vedtext{āgneyāṃ}{
	\variants{\emd ; agneyāṃ \MSN}
} \vedtext{karañjavṛkṣe}{
	\variants{\emd ; karañja \MSN}
} vaiśvānaraḥ śuklavarṇaś chāgānanaḥ.
latājaṭyāṃ\footnote{
	\begin{english}%
		Other sources name this tree as \emph{parkaṭī}, which Amara gives \emph{jaṭī} as a synonym.
	\end{english}
} naravāhano\footnote{
	Amara record \emph{naravāhana} as a name for Kubera, but here it should be understood as referring to Nairṛti/ Nirṛti, who indeed generally has a human as his mount (\cite[98 ff.]{wesselsmevissen2001}). \TIB\ appears to have mistranslated this sentences, perhaps being thrown off by the direction being mentioned at the end of the sentence: \emph{bden bral du la da dza ba ḍu'i shing dang/ mi'i gdong} [D f. 220r1] \emph{can dang/ bden bral dkar po'o/ /} (\emph{du la da dza ka ba ṭu'i}] P N; \emph{dul da dza ba ḍu'i} D).
} manuṣyamukhaḥ pāṇḍur nairṛtyām.
vāyavyāṃ kakubhavṛkṣe pavano mṛgānanaḥ pītaḥ.
\vedtext{aiśānyāṃ}{
	\variants{\emd ; aiśānyā \MSN}
} bhūteśo vṛṣabhānanaś citro nyagrodhapādape.\footnote{
	\TIB\ rendering \emph{bhūteṣa} as \emph{'byung po} rather where one might expect \emph{'byung po'i dbang po}. It has also construed \emph{citra} as qualifying \emph{nyagrodhapādapa} and adds a finite verb: \emph{dbang ldan gyi mtshams su 'byung po dang/ khyu mchog gi gdong can no/ /nya gro dha sna tshogs pa bsgom mo/ /}
}
sarve cāmī vāmakarakalitakapālā nānāstravyagradakṣiṇapāṇayo darśitapūrvārdhakāyāḥ.
\pend

% Amara: bodhivṛkṣa = aśvattha
% Amara: kakubha = arjuna
% Amara: nyagrodha = vaṭa
% vaiśvānaraḥ = Agni
% Naravāhana = Kubera

% southeast, southwest, northwest, northeast
% de nas rdo rje ra ba'i nang rol du dur khrod brgyad bsam par bya'o/ /shar phyogs su dbang po dang/ aa ru ra'i shing dang/ glang po che'i gdong can ser po'o/ /
% lho phyogs su gshin rje dang/ tsu ta'i shing dang/ ma he'i gdong can kha dog dkar po'o/ /
% nub phyogs su aa sho ka'i shing dang/ chu lha kha dog dmar ba dang/ seng ge'i gdong can no/ /
% byang phyogs su byang chub kyi shing dang/ lus ngan ser po dang/ /mi'i gdong can no/ /

% me'i mtshams su shing ka ranydza dang/ me lha dkar po ra'i gdong pa can no/ /
% bden bral dul da dza ba du'i shing dang/ mi'i gdong [D f. 220r] */ /can dang/ bden bral dkar po'o/ /
% rlung gi mtshams su ka ku pa'i shing dang/ rlung lha dang/ ri dags kyi gdong can ser po'o/ /
% dbang ldan gyi mtshams su 'byung po dang/ khyu mchog gi gdong can no/ /n+ya gro dha sna tshogs pa bsgom mo/ /
% 'di dag thams cad lag pa g-yon par thod pa 'dzin pa/ phyag g-yas par rang rang gi mtshon cha sna tshogs 'dzin pa'o/ /lus kyi stod bstan pa'o/ /

\pstart
evaṃ pūrvādyaṣṭadikṣu yathākramam ananta\dsh padma\dsh vāsuki\dsh mahāpadma\dsh takṣaka\dsh śaṃkhapāla\dsh karkoṭa\dsh kulikāḥ.
meghāś cāṣṭau mecaka\dsh śukla\dsh śiti\dsh pāṇḍu\dsh rakta\dsh pīta\dsh harita\dsh viśvavarṇāś\footnote{
	\begin{english}%
		As mentioned above, \TIB\ renders \emph{mecaka} as \emph{ser po}.
		For \emph{śukla} it reads \emph{skya bo}, whereas elsewhere in the translation \emph{śukla} is rendered as \emph{dkar po}.
		The Sanskrit \emph{pāṇḍu} is rendered well here as \emph{dkar ser}, but above it had been rendered simply as \emph{dkar po}.
		The colours Advayavajra gives for the clouds match exactly those given in the \emph{Aṣṭaśmaśāna}.
	\end{english}
} cintanīyāḥ.\footnote{\begin{english}
	The word \emph{cintanīyāḥ} has no reflex in \TIB .
\end{english}}
evaṃ dikpālāś \vedtext{cāśṭau [prasiddhavarṇāḥ.
aṣṭa caityāny api sa]pakṣavarṇāni}{
	\lemma{cāṣṭau prasiddhavarṇā \ldots\ sapakṣavarṇāni} 
	\variants{\diag ; cāṣṭau pakṣavarṇṇāni \MSN ; de ltar phyogs skyong brgyad kyi kha dog ni grags par zad do/ /mchod rten brgyad kyi kha dog kyang rtogs par bya'o/ / \TIB\ (starting from \emph{evaṃ dikpālāś} to \emph{bodhavyāni})}
} bodhavyāni.\footnote{\begin{english}
	Here again it is evident that the transmitted text, which reads \emph{evaṃ dikpālāś cāṣṭau pakṣavarṇṇāni bodhavyāni}, has suffered from an eye skip.
	The neuter form \emph{°varṇāni} lends support to what can be understood from \TIB : namely, that there should be a second sentence regarding the colour of \emph{caitya}s.
	The first sentence, which in \TIB\ reads \emph{de ltar phyogs skyong brgyad kyi kha dog ni grags par zad do}, is relatively unproblematic.
	Here I have `back translated' \emph{grags par zad do} with \emph{prasiddha}—attestations for this correspondence cannot be found in the translation of mTshur Ye shes 'byung gnas, but we do find attestations elsewhere, such as in 'Gos lhas brtsas's translation of Ratnākaraśānti's MuĀ ad HeTa 2.4.53 (ed.\ p.\ 183; D f.\ 186r).

	The second sentence is slightly more problematic. \TIB 's reading—\emph{mchod rten brgyad kyi kha dog kyang rtogs par bya'o}—suggests something along the lines of, \emph{aṣṭacaityānāṃ varṇā api bodhavyāḥ}.
	Our Sanskrit manuscript, on the other hand, indicates that the sentence ends with \emph{pakṣavarṇāni bodhavyāni}.
	I suspect, therefore, that, for one reason or another, words are missing from the Tibetan translation, which is indeed overly terse.
	Perhaps it could have read: \emph{mchod rten brgyad kyi kha dog kyang de dang mthun par rtogs par bya'o/ /}.
	In any case, the compound \emph{sapakṣavarṇa} is, as far as I can tell, unattested elsewhere, so perhaps it is not the best conjecture; nevertheless, because it does seem to make good sense in the context, and because it requires minimal alterations to the transmitted reading, I provisionally propose it as currently the best solution. 
\end{english}}
\pend

% de bzhin du shar phyogs la sogs pa tshogs brgyad du go rims ji lta ba bzhin du/ mtha' yas dang/ padma dang/ nor rgyas dang/ padma chen po dang/ 'jog po dang/ dung skyong dang/ karko ta dang/ rigs ldan rnams so/ /sprin brgyad ni/ ser po dang/ skya bo dang/ dkar po dang/ skya ser dang/ dmar po dang/ ser po dang/ ljang gu dang/ kha dog sna tshogs pa rnams so/ /de ltar phyogs skyong brgyad kyi kha dog ni grags par zad do/ /mchod rten brgyad kyi kha dog kyang rtogs par bya'o/ /

\pstart
tato maṇḍalamadhye 'ṣṭadalaṃ raktakamalaṃ vicintya, tatkamalamadhyapūrvādicaturdaleṣu\footnote{\begin{english}
	\TIB\ lacks a reflex of \emph{tat} in \emph{tatkamala°}: \emph{padma'i nang shar phyogs la sogs pa'i 'dab ma bzhi}.
\end{english}} tathā kūṭāgāracatuḥkoṇeṣu caturdvāreṣv adha ūrdhvaṃ ca pañcadaśaśavān\footnote{\begin{english}
	\TIB\ erroneously reads \emph{ro bco lnga dang ldan par} for \emph{pañcadaśaśavā}.
\end{english}} paśyet. 
tadanantaraṃ śavārūḍhān ālikālipariṇatacandrasūryamadhyagatān akārādipañcadaśasvarān dhyāyāt.
tataḥ \begin{mantra}a\end{mantra}\dsh kārādibījapariṇāmena sarvatra sabījā kartikā.
tataś candrasūryasabījakartikā\fnum{\MSN}{262}{r}pariṇāmena\footnote{\begin{english}
	The compound \emph{candrasūryasabījakartikāpariṇāma} as rendered in \TIB\ does not reflect \emph{sabīja}: \emph{zla ba dang/ nyi ma dang/ gri gug yongs su gyur pa las}.
\end{english}} nairātmyādipañcadaśa\vedtext{yoginīr}{
	\lemma{°yoginīr}
	\variants{\emd ; °yoginī \MSN}
} dhyāyāt. 
tatrādarśajñānavāṃś candraḥ, samatājñānavān\footnote{\begin{english}
	\TIB\ lacks a reflex for \emph{jñāna} in \emph{samatājñānavān}: \emph{mnyam pa nyid dang ldan pa}. 
\end{english}} sūryaḥ, tayor madhyagataṃ bījaṃ pratyavekṣaṇā, sarveṣām aikyaṃ kṛtyānuṣṭhānam, bimbaniṣpattiḥ suviśuddhadharmadhātuḥ.\footnote{\begin{english}
	This last sentence is a paraphrase of HeTa 1.8.6c–7: \emph{ādarśajñānavāṃś candraḥ samatāvān saptasaptikaḥ || bījaiś cihnaṃ svadeavsya pratyavekṣaṇam ucyate | sarvair aikyam anusṭhānaṃ niṣpattiḥ śuddhidharmatā ||} 
\end{english}}
\pend

% TODO: Update HeTa verses

% de nas dkyil 'khor gyi 'khor lo'i nang du padma 'dab ma brgyad pa dmar po bsams te/ padma'i nang shar phyogs la sogs pa'i 'dab ma bzhi dang/ de bzhin du gzhal yas khang gi zur bzhi dang/ sgo bzhi dang/ steng dang 'og rnams su ro bco lnga dang ldan par blta bar bya'o/ /

% de'i rjes su ro mnan pa'i a'a li dang k'a li yongs su gyur pa las/ zla ba dang nyi ma'i nang du son pa'i aa la sogs pa'i dbyangs bco lnga bsgom par bya'o/ /de nas aa la sogs pa'i yi ge yongs su gyur pa las thams cad (las)(perhaps read la?) sa bon dang bcas pa'i gri gug go/ /de nas zla ba dang/ nyi ma dang/ gri gug yongs su gyur pa las bdag med ma la sogs pa rnal 'byor ma bco lnga bsgom par bya'o/ /

% de la me long lta bu'i ye shes dang ldan pa ni zla ba'o/ /mnyam pa nyid dang ldan pa ni nyi ma'o/ /de dag gi nang du (son pa)(sic for son pa'i sa bon?) ni so sor rtog pa'o/ /thams cad gcig pa ni bya ba nan tan no/ /gzugs brnyan rdzogs pa ni chos kyi dbyings shin tu rnam par dag pa'o/ /

\pstart
atra varaṭakamadhye dhyeyāḥ \vedtext{\begin{mantra}a\end{mantra}\dsh kārasvarasambhavā}{
	\variants{\emd ; akārasvarasahyā \MSN}
} dveṣātmikākṣobhyamudritā vijñānaskandhātmikā prajñopāyasvarūpā bahirupāyarūpakhaṭvāṅgāliṅgitakandharā\footnote{\begin{english}
	\TIB\ may reflect a different reading for \emph{bahirupāyarūpakhaṭvāṅgāliṅgitakandharā}: `\emph{kha tvaṃ gas gzugs 'khyud pa'o}'.
	There is no reflex of \emph{bahirupāya°}, and \emph{gzugs} may be based on a Sanskrit word other than \emph{kandhara}, which the translator previously rendered as \emph{mgul}.
\end{english}} nairātmyā.
pūrvādidaleṣu \begin{mantra}\vedtext{ā\dsh i\dsh ī}{
	\variants{\emd ; a ā i \MSN}
}\dsh u\end{mantra}\dsh svarasambhavā mohapaiśunya\vedtext{rāgerṣyāsvabhāvā}{
	\lemma{°rāgerṣyāsvabhāvā}
	\variants{\emd ; °rāgairṣyāsvabhāva \MSN}
} vairocanaratnasambhavāmitābhāmoghasiddhimudritā rūpavedanāsaṃjñā\vedtext{saṃskāra}{
	\lemma{°saṃskāra°}
	\variants{\emd ; °saṃskārā° \MSN}
}skandhātmikā vajrāgaurīvārīvajra\vedtext{ḍākinīr}{
	\lemma{°ḍākinīr}
	\variants{\emd ; °ḍākinī \MSN}
} dhyāyāt. 
\pend

% TODO: recheck rāgerṣyāsvabhāvā 

% de lta lte ba'i nang du dbyangs kyi yi ge aa las byung ba'i mi bskyod pas rgyas btab pa/ zhe sdang dang/ rnam par shes pa'i phung po'i bdag nyid thabs dang shes rab kyi ngo bo'i gzugs kha tv'am gas gzugs [D f. 220v] 'khyud pa'o/ /

% shar la sogs pa'i 'dab ma rnams la/ aa ai a'i a'u'i dbyangs las byung ba/ gti mug dang/ phra ma dang/ 'dod chags dang/ phrag dog gi rang bzhin/ rnam par snang mdzad dang/ rin chen 'byung ldan dang/ 'od dpag med dang/ don yod grub pa rnams kyis rgyas btab pa/ gzugs dang/ tshor ba dang/ 'du shes dang/ 'du byed kyi phung po'i bdag nyid/ rdo rje ma dang/ goo r'i ma dang/ chu ma dang/ rdo rje mkha' 'gro ma rnams bsgom par bya'o/ /

\pstart
tato bāhyapuṭe aiśānyādikoṇeṣu \begin{mantra}\vedtext{ū}{
	\variants{\emd ; ṛ ū \MSN}
}\dsh ṛ\dsh ṝ\dsh ḷ\end{mantra}\dsh \vedtext{svaraniṣpannā}{
	\lemma{°svaraniṣpannā}
	\variants{\emd ; °svarāni<ā>ṣpannā \MSN}
}\footnote{\begin{english}
	\TIB\ erroneously reads \emph{dbyangs la sogs pa'i} for \emph{svaraniṣpannāḥ}. 
\end{english}} akṣobhyavairocana\vedtext{ratnasambhavāmitābha}{
	\lemma{°ratnasambhavāmitābha°}
	\variants{\emd; °ratnasambha((ḥ))vāmitābhā° \MSN}
}mudritāḥ pṛthivyāptejovāyusvabhāvāḥ pukkasīśabarīcaṇḍālīḍombīḥ paśyet. 
\pend

% de nas phyi rol gyi rim pa la dbang po la sogs pa'i mtshams rnams su a'u ri r'i li dbyangs la sogs pa'i mi bskyod pa dang/ rnam par snang mdzad dang/ rin chen 'byung ldan dang/ 'od dpag med pas rgyas btab pa/ sa dang/ chu dang/ me dang/ rlung rnams kyi rang bzhin/ pukka s'i dang/ sha ba r'i dang/ tsandal'i dang/ dombi ma rnams bsgom par bya'o/ /

\pstart
tataḥ pūrvādidvāreṣu \begin{mantra}ḹ\dsh e\dsh ai\dsh o\end{mantra}\dsh svarasambhavāḥ sparśādimudrāmudritā\footnote{\begin{english}
	For `\emph{sparśādimudrāmudritāḥ}' \TIB\ reads `\emph{pukka sī la sogs pa'i rang bzhin du rgyas btab pa}', 
	\hl{TODO}: What could have caused this? See HeTa 2.4.18: \emph{gaurīñ ca dveṣataḥ| caurīṃ mohamudreṇa vetālīṃ piśunamudrayā| ghasmarīṃ rāgamudreṇa}: i.e. Akṣobhya, Vairocana, Ratnasaṃbhava, Amitābha.
	So maybe conjecture \emph{pukkasyādimudrāmudritā} - `they are sealed by the [same] seals of Pukkasī and so on'?
\end{english}} \vedtext{rūpaśabdagandharasa}{
	\lemma{rūpaśabdagandharasa°}
	\variants{\emd\ (\TIB\ gzugs dang/ sgra dang/ dri dang/ ro rnams); rūpaśabdagandha° \MSN}
}svarūpā gaurīcaurīvettālīghasmaryo bhāvyāḥ. 
\vedtext{tadanantaraṃ}{
	\variants{\emd ; tadanantara \MSN}
} adha ūrdhvaṃ ca \vedtext{moharāgamudrite}{
	\variants{\emd\ (\TIB\ gti mug dang 'dod chags kyis rgyas btab pa); moharāge mudrite \MSN}
} \begin{mantra}au\dsh aṃ\end{mantra}\dsh svarasambhave sparśadharmadhātusvabhāve bhavanirvāṇasvarūpe bhūcarī\vedtext{khecaryau}{
	\lemma{°khecaryau}
	\variants{\emd ; °khecaryyā \MSN}
} bhāvayet.
\pend

% de nas shar phyogs la sogs pa'i sgo rnams su/ l'i ae aee ao'i dbyangs las byung ba pukka s'i la sogs pa'i rang bzhin du rgyas btab pa/ gzugs dang/ sgra dang/ dri dang/ ro rnams kyi rang bzhin/ goo r'i dang/ tsoo r'i dang/ be t'a l'i dang/ ghasma r'i rnams bsgom par bya'o/ /
% de'i rjes su steng dang 'og tu gti mug dang 'dod chags kyis rgyas btab pa/ aoo aam las byung ba/ reg bya dang/ chos kyi dbyings kyi rang bzhin te/ 'khor ba dang mya ngan las 'das pa'i rang bzhin sa spyod dang/ mkha' spyod dag bsgom par bya'o/ /

\pstart
atra ca devīnām utpatyanantaraṃ \vedtext{svakuleśā}{
	\lemma{svakuleśā°}
	\variants{\MSN\PCreading ; svakraleśā° \MSN\ACreading}
}bhiṣeke sati svakuleśamudrā bodha\fnum{\MSN}{262}{v}vyā. 
\pend

% 'dir yang lha mo rnams bskyed pa'i rjes su rang gi rigs dang bcas pa'i dbang bskur te/ rang gi rigs kyis rgyas btab par rtogs par bya'o/ /

\pstart
etā pañcadaśayoginyaḥ ṣoḍaśābdāḥ sūryamaṇḍalasthā bhinnāñjanābhā\footnote{\begin{english}
	\TIB\ erroneously renders \emph{bhinnāñjanābhā} as \emph{dbyer med pa'i mig sman nag po'i mdog lta bu}, as if reading \emph{abinnāñjanābhā}.
	A more common rendering by Tibetan translators for \emph{bhinnāñjana} is \emph{stang zil bcag pa} (see, for example, the \emph{Lalitavistara}, in prose after 7.54 and D f.\ 57r). TODO: reference.
	On the meaning of the compound and examples from literature, see \cite{vogel1968}.
\end{english}} bodhicittasvabhāvā jvalitapiṅgalordhvakeśās \crux tāluke\crux\ \vedtext{vajrasattvasvabhāva}{
	\lemma{vajrasattvasvabhāva°}
	\variants{\emd\ (\TIB\ rdo rje sems dpa'i rang bzhin thod pa); vajrasatvasvabhāvā \MSN}
}caturaṅgulakapāladhāriṇyaḥ śirasi ca pañcabuddhasvabhāva\vedtext{śuddhāni}{
	\lemma{°śuddhāni}
	\variants{\conj ; śuddha \MSN ; \emph{no reflex in }\TIB .}
}\footnote{\begin{english}
	The manuscript's reading of \emph{pañcabuddhasvabhāvaśuddhapañcamuṇḍāni} yields an unnatural \emph{karmadhāraya} compound.
	\TIB\ does not have any word corresponding to \emph{śuddha}, so one may also consider the conjecture `\emph{pañcabuddhasvabhāvāni pañca muṇḍāni}'.
	I believe both conjectures are equally plausible.
\end{english}} pañca muṇḍāni vibhratyo raktavartulatrinetrā daṃṣṭrākarālavadanāḥ \vedtext{pañcadaśamātṛkāsvabhāvaśuṣka}{
	\lemma{pañcadaśamātṛkāsvabhāvaśuṣka°}
	\variants{\emd\ (\TIB\ ma mo bco lnga'i bdag nyid kyi mgo bo skam po); pañcadaśamātṛkāsvabhāvā śuṣka° \MSN}
}pañcadaśamuṇḍamālālaṅkṛtā vyāghra\vedtext{carmāvṛta}{
	\lemma{°carmāvṛta°}
	\variants{\emd ; °carmmāvṛtā \MSN}
}kaṭinitambā ardhaparyaṅkanāṭyasthāḥ śavārūḍhāḥ pañcamudrādharāḥ. tatra\Emdash
\pend

% bhinnāñjanābhā] ms; dbyer med pa'i mig sman nag po'i mdog lta bu D

% de dag ni rnal 'byor ma bco lnga'o/ /lo bcu drug pa'i tshul zla ba'i dkyil 'khor la gnas pa'i dbyer med pa'i mig sman nag po'i mdog lta bu ste/ byang chub sems kyi rang bzhin skra 'bar ba gyen du brdzes pa/ spyi bor rdo rje sems dpa'i rang bzhin thod pa dum bu sor bzhi pa 'dzin pa/ mgo la yang sangs rgyas lnga'i rang bzhin thod pa skam po lnga 'dzin pa'o/ /spyan zlum po dmar po gsum pa zhal mche ba gtsigs pa/ ma mo bco lnga'i bdag nyid kyi mgo bo skam po bco lnga'i phreng bas brgyan pa/ stag gi pags pa'i sham thabs can/ skyil krung phyed pa'i gar gyis [D f. 221r]*/ /bzhugs pa ro la bcibs pa/ phyag rgya lnga 'dzin pa'o/ /

\medskip\versequote
akṣobhyaś cakrirūpeṇ\vedtext{āmitābhaḥ}{
	\lemma{°āmitābhaḥ}
	\variants{\emd ; °āmitābha \MSN}
} kuṇḍalātmakaḥ | &
ratneśaḥ kaṇṭhamālāyāṃ haste vairocanaḥ sthitaḥ\footnote{\begin{english}
	See note \ref{2bnote} on the reading \emph{sthitaḥ}.
\end{english}} || 1 ||\&

% de la/ 
% 'khor lor gzugs kyis mi bskyod pa/ /
% rna cha'i bdag nyid 'od dpag med/ /
% mgul pa'i phreng bar rin chen bdag /
% lag rgyan rnam par snang mdzad brjod/ /

\medskip\versequote
mekhalāyāṃ sthito 'moghaḥ sarvāṅge vajradhṛk tathā |\footnote{\begin{english}
	1-2a corresponds to HeTa 1.6.11–12a, and these \emph{pāda}s along with 2b are found in Saroruhavajra's \emph{Sādhanaopāyikā} (ed.\ p.\ 112) and Bhadrapāda's \emph{Dveṣavajrasādhana} (ed.\ p.\ 358).
	Given the lack of parallel material for 2b (another instance is \emph{Vajrācāryakriyāsaṅgraha} D f.\ 201r) (TODO reference), it is likely that Advayavajra is here drawing on the \emph{Sādhanopāyikā}.
	
	In 1d, \emph{smṛtaḥ} is the reading found in all palm-leaf witnesses of the HeTa available to me, and this is reflected in the Tibetan translations of the tantra (D f.\ 7r) and the NaiPra: \emph{rnam par snang mdzad brjod}.
	The reading is also found in the verses as they appears in the \emph{Sādhanamālā} edition of Ḍombīheruka's \emph{Amṛtaprabhā} (p.\ 447), \emph{Saṃpuṭatantra} 5.4.33d, and the \emph{Vajrāvalī} (ed.\ p.\ 452).
	Witnesses of both Saroruhavajra's \emph{Sādhanopāyikā} and Bahdrapāda's \emph{Dveṣavajrasādhana}, however, all support \emph{sthitaḥ}.
	I therefore believe that the reading \emph{sthitaḥ} should be maintained in the NaiPra, despite the Tibetan translation, which here appears to have been influenced by 'Brog mi's translation of the root tantra.

	The word \emph{tathā} in 2b is supported by \TIB : \emph{lus kun rdo rje 'dzin bzhin no}.
	For the parallels, Gerloff's edition of the \emph{Sādhanopāyika} prints \emph{vaset}, which is reported to be found in one paper witness, while \emph{paśyet}, unmetrical and ungrammatical, is the reading of the \emph{Hevajrasādhanasaṅgraha} codex.
	The canonical Tibetan translation here reportedly reads \emph{yan lag kun spyod rdo rje 'dzin} (\cite[vol.\ 1 135]{gerloff2020}), while a para-canonical translation reads \emph{yan lag kun la rdor 'dzin dgod} (\cite[vol.\ 2 152]{gerloff2020})—neither, I believe, clearly favours any of the available Sanskrit readings.
	As reported by Gerloff, the \emph{Hevajrasādhanasaṅgraha} codex also reads \emph{paśyet} in the \emph{Dveṣavajrasādhana}, and this is presently the text's sole witness (there is no known Tibetan translation).
	In sum, I think it is difficult to regard any of the possible readings for the final word of 2b as particularly secure.
	\label{2bnote}
\end{english}} & 
gurvācāryeṣṭa\vedtext{devasya}{
	\lemma{°devasya}
	\variants{\emd ; °devatāsya \MSN}
} namanāya śirasi cakrikā || 2 ||\&

% ske rags la ni don yod gnas/ / 
% lus kun rdo rje 'dzin bzhin no/ /
% bla ma slob dpon 'dod lha la/ /
% phyag 'tshal spyi bor 'khor lo 'dzin/ /

\medskip\versequote
durbhāṣasy\vedtext{āśravaṇāya}{
	\lemma{°āśravaṇāya}
	\variants{\emd\ (\TIB\ mi nyan pa); °āśramaṇāya \MSN}
} guror vajradharasya ca | &
karṇayoḥ kuṇḍalaṃ \vedtext{dhāryaṃ}{
	\variants{\emd ; dhāryya \MSN}
} mantrajāpāya kaṇṭhikā || 3 || \& 

% bla ma rdo rje 'dzin pa la'ang/ /
% smod tshig mi nyan pa yi phyir/ /
% rna ba dag la rna cha 'dzin/ /
% sngags bzlas phyir ni mgul ba'i phreng/ /

\medskip\versequote 
mekhalā bhajituṃ mudrāṃ tyaktuṃ prāṇivadhaṃ rucakaḥ |\footnote{\begin{english}
	2c–4b corresponds to HeTa 2.6.3–4d with slight variations.
	To asses these variations, we should first consider \TIB, which reads: 

	\begin{prosequote}%
		bla ma slob dpon 'dod lha la/ /\\
		phyag 'tshal spyi bor 'khor lo 'dzin/ /\\
		\\
		bla ma rdo rje 'dzin pa la'ang/ /\\
		smod tshig mi nyan pa yi phyir/ /\\
		rna ba dag la rna cha 'dzin/ /\\
		sngags bzlas phyir ni mgul ba'i phreng/ /\\
		\\
		phyag rgya bsten pa ske rags te/ /\\
		srog gcod spangs pa gdu bu ste/ /
	\end{prosequote}

	The verses in the HeTa run as follows:

	\begin{prosequote}%
		gurvācāryeṣṭadevasya namanārthaṃ cakrikā dhṛtā |\\
		durbhāṣasyāśravaṇāya guror vajradharasya ca || 2.6.3 ||
		\medskip	

		\noindent \textbf{3b} namanārtha] \sigmareading{P}; navanārthaṃ P
		\textbf{3b} cakrikā] \sigmareading{C}; cakṛkā C
		\textbf{3c} °āśravaṇāya] \sigmareading{P}; °āśramaṇāya P
		\textbf{3d} guror] \sigmareading{K}; guro K\\

		\noindent śravaṇayoḥ kuṇḍalaṃ dhāryaṃ mantraṃ japtuṃ ca kaṇṭhikā |\\
		rucakaḥ prāṇivadhaṃ tyaktuṃ mudrā bhajituṃ ca mekhalaṃ | 2.6.4a–d
		\medskip

		\noindent \textbf{4a} kuṇḍalaṃ dhāryaṃ] \sigmareading{K}; kuṇḍalaṃ dhāryya K
		\textbf{4b} japtuṃ] \sigmareading{K}; japtaṃ K
		\textbf{4c} rucakaḥ] C N K; rucakaṃ P E
		\textbf{4c} prāṇivadhaṃ] \sigmareading{E}; prāṇivandhaṃ E
		\textbf{4d} mudrā] \sigmareading{E}; mudrāṃ E
		\textbf{4d} bhajituṃ] C\PCreading\ N (bhajituñ) E; bhañjituṃ P K (bhaṃjituñ)
	\end{prosequote}

	At present I am uncertain why we see these variations.
	One possibility is that Advayavajra simply used material from the tantra imprecisely.
	Of the variations, \TIB\ includes a reflex of \emph{śirasi} in 2d (= HeTa 2.6.3b), a word that we have no evidence for in the HeTa.
	It also supports reversing \emph{pāda}s c and d of HeTa 2.6.4.
	\TIB\ does not clearly offer support for or against the remaning variants. 
	Of these, the reading \emph{°devatāsya} for 2c (= HeTa 2.6.3a) found in the NaiPra's ms is an impossible form and metrically bad, and thus it must be rejected.
	It is interesting that the NaiPra's ms reading of \emph{aśramaṇāya} in 3a (= HeTa 2.6.3c) is supported by a palm-leaf witness of the HeTa, but this does not make very good sense and should also probably be rejected.
	The variants \emph{namanārthaṃ} v.\ \emph{namanāya} in 2d (= HeTa 2.6.3b) and \emph{japtum} v.\ \emph{°jāpāya} in 3d (= HeTa2.6.4b) are equivalents, and I see no way of easily determining which of these forms Advayavajra originally wrote.
	The replacement of \emph{śravaṇayoḥ} with \emph{karṇayoḥ} results in a slight metrical improvement, but it must be said that imposing stricter metre on these verses was evidently not a priority for Advayavajra here. 
	Finally while the tantra's manuscripts appear to point towards reading \emph{mudrā} (probably to be understood as \emph{mudrāḥ}, accusaitve plural), it seems equally possible that Advayavajra wrote \emph{mudrām}, as the NaiPra's ms indicates.
	%Tib reads: bla ma slob dpon 'dod lha la/ /phyag 'tshal don du 'khor lo 'dzin/ /bla ma rdo rje 'dzin pa la/ /smod tshig mi nyan pa yi phyir/ /rna ba dag tu rna cha 'dzin/ /sngags kyi bzlas pa mgul rgyan nyid/ /gdu bu srog chags gsod pa spangs/ /phyag rgya bsten pa ska rags nyid/ /.
\end{english}} &
nūpurakeyūradharāḥ kṛṣnāṅgo maitracittataḥ || 4 ||\footnote{\begin{english}
	4d corresponds to HeTa 2.9.11b.
	Note that Snellgrove's edition reads \emph{maitricittataḥ}, with no variants reported.
	The more expected \emph{maitrīcittataḥ} would be unmetrical, but all palm-leaf manuscripts of the tantra that are available to me support \emph{maitracittataḥ}, as do citations of the \emph{pāda} in Ratnākaraśānti's \emph{Bhramaharasādhana} (ed.\ p.\ 166) and the \emph{Sādhanopāyikā} (ed.\ p.\ 111). 
\end{english}} \&

% phyag rgya bsten pa ske rags te/ /
% srog gcod spangs pa gdu bu ste/ /
% rkang rgyan dang ni dpung rgyan 'dzin/ /
% byams pa'i thugs ni yan lag gnag /

\pstart
keśānāṃ raktapiṅgatā mahārāgatākhyāpanāya, krodhapratipādanāyordhvatā.\footnote{\begin{english}
	\TIB\ treats this sentence as though it were verse.
\end{english}}
kāyavākcetasām atirāgasvabhāvatvāt \vedtext{svabhāvena}{
	\variants{\conj\ (\TIB\ rang bzhin gyis) ; svabhāvānāṃ \MSN ; tatsvābhāvānāṃ \possibleconj}
} netrāṇāṃ mahārāgatā.
bhavanirvāṇasvabhāvau bāhū.
mānādidoṣān kartituṃ kartikā.\footnote{\begin{english}
	cf.\ HeTa 1.8.20a: \emph{tathā mānādiṣaḍdoṣān kartituṃ kartikā} (note that Snellgrove's edition reads \emph{kartṛkā}).
\end{english}}
traidhātukaviśuddhyā skandhādicaturmārarudhirapūrṇaṃ \vedtext{trikhaṇḍaṃ}{
	\variants{\MSN\PCreading ; trimukhaṇḍaṃ \MSN\ACreading}
} \vedtext{sakalavi\fnum{\MSN}{263}{r}kalpaśarīri}{
	\lemma{sakalavikalpaśarīri}
	\variants{\emd\ (\TIB\ rnam par rtog pa mtha' dag gis lus); kamalavikalpaśarīri \MSN}
} kapālam.
dharmasambhoganirmāṇaviśuddhyā \vedtext{tribhaṅgaḥ}{
	\variants{\emd ; tribhaṅgāṃ \MSN}
}. % Hevajraviśudhinisādhana tribhagaṃ (f. 77v 6); see also Vajrāvalī. Tib.\ sum khugs.
svābhāvikakāyaviśuddhyā śarīrayaṣṭiḥ.
anāvaraṇatākhyāpanāya vyāghracarmavasanatā.
traidhātukānālambanatākhyāpanāy\crux \vedtext{ānardhenāka}{
	\lemma{°ānardhenāka}
	\variants{\MSN ; zhabs ma sbyangs pa'o \TIB\ (\emph{sic for} ma brkyangs pa?); \emph{cf. Hevajraviśuddhinisādhana }fol.\ 77v6–7: sakalatraidhātukanirālambaviśuddyā ardhaparyaṅkatā}
}\crux caraṇatā.
ekarasatākhyāpanāyaikapādākrāntabhūtalatā.\footnote{\begin{english}
	\TIB\ perhaps reflects \emph{ekarasasvabhāvatā} in place of \emph{ekarasatā}: \emph{ro gcig pa'i rang bzhin}.
\end{english}}
\pend

% kamalavikalpaśarīri] ms; sakalavikalpaśarīri em. (D: rnam par rtog pa mtha' dag gis lus)  

% dmar ser gyi ni skra dag ni/ /
% 'dod chags chen por ston pa dang/ /
% khro bor rtogs phyir steng du brdzis/ / (probably not verse in Skt.)
% sku dang gsung dang thugs shin tu chags pa'i rang bzhin gyi phyir/ rang bzhin gyis spyan dmar po chen po'o/ /
% srid pa dang mya ngan las 'das pa'i rang bzhin dag ni phyag dag go/ /
% nga rgyal la sogs pa'i skyon gcod pa ni gri gug go/ /
% khams gsum pa rnam par dag pa phung po la sogs pa'i bdud bzhi la sogs pa'i khrag gis bkang ba rnam par rtog pa mtha' dag gis lus thod pa dum bu gsum mo/ /
% chos dang longs spyod dang sprul pa rnam par dag pa ni gnyer ma gsum mo/ /
% ngo bo nyid gcig pu'i sku'i rnam par dag pa ni lus kyi ril po dag go/ /
% sgrib pa dang bral bar ston pa'i phyir stag gi pags pa'i na bza'o/ /
% khams gsum pa dmigs pa med par ston pa'i phyir zhabs ma sbyangs pa'o/ /
% ro gcig pa'i rang bzhin du bstan pa'i phyir zhabs gcig gis sa'i dkyil 'khor du mnan pa'o/ /

\pstart
tadanantaraṃ hṛdvartibīja\vedtext{vinirgataiḥ}{
	\lemma{°vinirgataiḥ}
	\variants{\emd ; °vinirgati \MSN}
} pañcākāraraśmibhir akaniṣṭhabhuvanavartijñānasattvasvabhāvaṃ nairātmyācakram ānīya hṛdbīje praveśayet.
jñānasattvasamayasattvayor aikyaṃ \vedtext{kṛtvā}{
	\variants{\emd\ (\TIB byas nas); bhūtvā \MSN}
} nairātmyāhaṃkāram udvahan \vedtext{nairātmyāsamo}{
	\variants{\emd\ (\TIB\ bdag med ma dang mnyam par); nairātmyāsamayo \MSN}
}\footnote{\begin{english}
	I believe we must accept the emendation of \emph{nairātmyāsamayaḥ} to \emph{nairātmyāsamaḥ} not simply because of support from \TIB , but also because it makes the most sense.
	The practitioner is already \emph{nairātmyāsamaya} insofar as he has been visualising himself as the goddess; however, only after dissolving the \emph{jñānasattva} into that visualisation does he become \emph{nairātmyāsama}.
\end{english}} bhavet. 
\pend

% de'i rjes su snying ga'i sa bon las byung ba'i 'od zer rnam pa lngas 'og min na bzhugs pa'i ye shes sems dpa'i rang gi bdag med ma'i 'khor lo la bkug la/ /snying ga'i sa bon la gzhug par bya'o/ /ye shes dang dam tshig sems dpa' dag gcig tu byas nas bdag med ma'i nga rgyal dang ldan pas bdag med ma dang mnyam par 'gyur te/ 

\pstart
atra ca ṣaḍaṅgayogavyavasthārtham anukrameṇa kṛṣṇarakta\vedtext{pītaharitanīla}{
	\lemma{°pītaharitanīla°}
	\variants{\emd\ (\TIB\ ser po dang/ ljang gu dang/ sngon po dang/); °pītaharitapītanīla° \MSN}
}śuklavarṇā bhāvanīyāḥ.\footnote{\begin{english}
	cf.\ HeTa 1.8.22c–24a: \emph{prathame bhāvayet kṛṣṇāṃ dvitīye raktām eva ca} || 22 || \emph{tṛtīye bhāvayet pītāṃ caturthe haritāṃ tathā | pañcame nīlavarṇāṃ ca ṣaṣṭame śukladehikām }|| (23) \emph{ṣaḍaṅgaṃ bhāvayed yogī |} (as printed in Snellgrove's edition, with orthographic normalisation).
\end{english}} 
\pend

% 'dir sbyor ba'i yan lag drug rnam par gzhag pa'i phyir rim pas/ [D f. 221v] nag po dang/ dmar po dang/ ser po dang/ ljang gu dang/ sngon po dang/ dkar po rnams bsgom par bya'o/ /


\pstart
tatra bhāvanāprakarṣaprakrameṇa prathamaṃ meghasaṃchannaṃ pūrṇacandravad bhāti.
tato 'pi prakarṣān māyāvad bhāti.
tato 'pi prakarṣāt svapnavat prakāśate.
tadanantaraṃ prakarṣaparipākāt svapnajāgraddaśayor abhedaprāpto mahāmudrāyogī sidhyati.
ity utpattikramaḥ. 
\pend

% de la bsgom pa phul du byung ba'i rim gyis dang por sprin gyis bsgribs pa'i zla ba'i dkyil 'khor lta bur snang ngo/ /de yang rab kyi phul du phyin pa las ni sgyu ma lta bur snang ngo/ /de yang rab kyi phul du phyin pa las ni rmi lam lta bur snang ngo/ /de'i rjes su rab kyi phul du phyin pa yongs su smin pa las ni rmi lam dang sad pa 'dra ba'i dus su gnyis su med par gang gis mthong bas phyag rgya chen po thob par 'gyur ro zhes bya bas ni bskyed pa'i rim pa'o/ /


\pstart
anyatra\footnote{\begin{english}
	\textcites[374]{mathes2014}[132]{mathes2021}, in an effort to advance his thesis that Advayavajra advocates a non-tantric form of Mahāmudrā practice, has written the following about this passage: `... it is not completely out of the question that an empowerment in Maitrīpa's system could start directly with the \emph{prajñājñāna}-empowerment. In his \emph{Nairātmyāprakāśa}, Maitrīpa thus explains the ordinary creation stage as an optional practice, and not as a necessary requirement for the subsequent stages.' 
	Judging by Mathes's translation, his assertion here appears to rest on having understood the word \emph{anyatra} in the sense of `alternatively'—perhaps as an equivilent to \emph{athavā}.
	I am unsure why we should understand \emph{anyatra} here as having a meaning other than the expected `elsewhere'.
	The meaning `elsewhere' makes good sense in the larger context of the \emph{Nairātmyāprakāśa}: the \emph{utpattikrama} taught here in this \emph{sādhana} consists in visualising the goddesses; elsewhere, a \emph{gambhīrotpattikrama} and forms of \emph{utpannakrama} are also taught, and they too are connected with Nairātmyā.

	Even were we to grant that Advayavajra intends \emph{anyatra} to mean `alternatively', this still does not support Mathes's interpretation that `the ordinary creation stage' is an optional practice.
	This would simply mean that practitioners can choose which practice to do—for instance, they may practice \emph{bāhyotpattikrama} on Mondays and \emph{gambhīrotpattikrama} on Tuesdays; but that does not tell us whether or not \emph{bāhyotpattikrama} is a necessary prerequisite for the subsequent stages.
	There may be evidence elsewhere in Advayavajra's corpus regarding this question, which is indeed a very interesting one, but here the matter is simply not addressed.
\end{english}} bolakakkolasaṃyogān mahāsukharūpi \vedtext{paramaviramamadhyagaṃ\footnote{\begin{english}
	\TIB\ renders \emph{virama} as \emph{bral ba}, which may be an acceptable translation but is at odds with the more common renderings of this technical term as either \emph{khyad par dga' ba} or \emph{dga' bral gyi dga' ba}.
\end{english}}}{
	\lemma{paramaviramamadhyagaṃ}
	\variants{\emd ; paramaviramadhyaga° \MSN}
} bodhicittaṃ jāyate yat tad eva pañcadaśakalātmakaṃ jhaṭiti pūrvoktavarṇacihnasaṃsthānapañcadaśa\vedtext{yoginīrūpaṃ}{
	\lemma{yoginīrūpaṃ}
	\variants{\emd ; yoginīrūpa \MSN}
} paśyet, \vedtext{tasya hi}{
	\variants{\MSN ; tasyāpi (\TIB\ de yang) \hl{\emph{possible em.}}}
} pañcaskandhacaturdhātuṣaḍviṣayakāyavākcittasvabhāvatvād iti \fnum{\MSN}{263}{v}gambhīrotpattikramaḥ.\footnote{\begin{english}
	This passage (beginning \emph{nairātmyāhaṃkāram udvahan}) has been translated in two publications by \textcites[373–4]{mathes2014}[132-3]{mathes2021}.
	In the former a draft edition of the passage by \textsf{Isaacson} is included in a footnote; and the latter publication also includes a translation of the sentence below that begins \emph{anābhogayuganaddhādvayavāhi}.
\end{english}}
\pend 

% gzhan du bo la dang kakkola dang yang dag par sbyor ba las/ gang bde ba chen po'i rang bzhin mchog dang bral ba'i nang du son pa'i byang chub kyi sems skye ste/ de nyid bco lnga'i cha'i bdag nyid can skad cig gis sngon du bstan pa'i phyag mtshan dang kha dog dang dbyibs te/ rnal 'byor ma bco lnga'i dkyil 'khor gyi 'khor lor blta bar bya'o/ /de yang phung po lnga dang/ khams bzhi dang/ yul drug ste/ sku dang gsung dang thugs kyi rang bzhin te/ 'di ni zab mo bskyed pa'i rim pa'o/ /


\pstart
jhagiti bījam \vedtext{anavalokayann}{
	\variants{\emd ; anavalokayana \MSN}
} eva pañcadaśayoginyātmakaṃ maṇḍalacakraṃ paśyed iti utpannakramaḥ. 
\pend

% skad cig nyid las rnal 'byor ma bco lnga'i bdag nyid kyi dkyil 'khor gyi 'khor lo blta'o/ /'di ni rdzogs pa'i rim pa'o/ /


\pstart
atha pariniṣpannakramaḥ.
vajraśarīre khalu jñānādhiṣṭhite\footnote{\begin{english}
	\TIB\ suggests reading \emph{pañcajñānādhiṣṭhitāḥ} as an adjective describing \emph{nāḍyaḥ}: \emph{ye shes lngas byin gyis brlabs pa'i rtsa}. 
	\MSN 's reading is slightly more convincing: that the body is presided over by \emph{jñāna} is frequently and famously expressed in the HeTa—for example, 1.1.12: \emph{dehastaṃ ca mahājñānam}.
	I also don't immediately see why the five forms of \emph{jñāna} need to be mentioned here.
\end{english}} \vedtext{dvātriṃśan}{
	\variants{\MSN\PCreading ; dvātriṃśatan \MSN\ACreading}
} nāḍyo mahāsukhasthānāt sravanti.
tāś ca pañcadaśa yoginya iti śarīram eva nairātmyācakrātmakam. 
tathā hi lalanārasane \vedtext{kaṇṭhād}{
	\variants{\emd\ (\TIB\ mgrin pa nas); karṇṇād \MSN}
} ārabhya nābhiṃ yāvad vāmetarapārśvavartinyo candrasūryākhye.
nābher adhas te eva yonināḍyau\footnote{\begin{english}
	For \emph{yonināḍyau}, \TIB\ erroneously reads \emph{skye gnas kyi rtsa la}, as if translating \emph{yonināḍyām}.
\end{english}} lalanārasane akṣobhyarudhiravahe.\footnote{\begin{english}
	This sentence strongly resembles a passage in Ratnākaraśānti's MuĀ ad HeTa 1.1.16:
	\emph{lalanārasane eva kaṇṭhād ārabhya yāvannābhiḥ.
	atrāntare vāmetarapārśvanāḍyau candrasūryākhye.
	nābher adhas te eva yonināḍyau lalanārasanākhye\footfoot{1} eva} (\footfoot{1}\emph{lalanārasanākhye}] ms-a ed.; \emph{lala...} ms-b [lost to damage]) (ms-a f.\ 17r; ms-c 12v; ed.\ p.\ 19).
	% MuĀ D: la la n'a dang/ ra sa n'a gnyis mgrin pa nas brtsams nas ji srid lte ba'i nang gi mthar thug pa ste/ g-yon pa dang g-yas pa'i phyogs kyi rtsa ni zla ba dang nyi ma zhes bya'o/ /lte ba'i 'og gi skye gnas kyi rtsa gnyis ni la la n'a dang ra sa n'a zhes bya ba de nyid do/ /
\end{english}}
avadhūtī śiraḥkaṇṭhahṛn\vedtext{nābhi}{
	\lemma{°nābhi°}
	\variants{\emd ; °nābhiṃ \MSN}
}yonimadhyasthā bodhicittāvahā.
etā nāḍyo nairātmyā. 
\pend

% rdo rje'i lus kyang ye shes lngas byin gyis brlabs pa'i rtsa sum cu rtsa gnyis te/ bde ba chen po'i gnas nas 'dzag pa'o/ /de yang rnal 'byor ma bco lnga'i bdag nyid de/ lus nyid bdag med ma'i 'khor lo'o/ /yongs su rdzogs pa'i rim pa'o/ /
% de bzhin dul la n'a dang ra sa n'a dag mgrin pa nas brtsams te/ lte ba'i bar du g-yon pa dang cig shos kyi ngos su gnas pa zla ba dang nyi ma zhes bya bas/ 
% lte ba'i nang du de nyid skye gnas kyi rtsa la la n'a dang ra sa n'a yin te/ mi bskyong pa dang khrag 'bab pa'o/ /aa ba dh'u t'i ni mgo dang mgrin pa dang lte ba dang skye gnas kyi nang du gnas pa byang chub kyi sems 'bab pa'o/ /rtsa de dag bdag med ma'o/ /


\pstart
abhedyāsūkṣme śiraḥśikhāsthe yathāsaṃkhyaṃ nakhadantakeśa\vedtext{roma}{
	\lemma{°roma°}
	\variants{\emph{orthographic change}; °loma° \MSN}
}lakṣaṇa\vedtext{yugala}{
	\lemma{°yugala°}
	\variants{\conj ; yugmayugala \MSN}
}vahe\footnote{\begin{english}
	\hl{QUESTION: Any justification for ms reading of \emph{yugmayugala}?}
\end{english}} vajrā. 
divyā dakṣiṇakarṇe \vedtext{tvaṅmalavahā}{
	\variants{\emd ; tvajmalavaho \MSN}
}, vāmā pṛṣṭhavaṃśe piśitavahā gaurī. 
vāmanīkūrmaje \vedtext{vāmakarṇabhrūmadhyasthe}{
	\variants{\emd ; vākarṇṇabhrūmadhyesthe \MSN}
} snāyvasthimālāvahe vārī. 
bhāvakīseke cakṣur\vedtext{bāhumūlasthe}{
	\lemma{°bāhumūlasthe}
	\variants{\emd ; °bāhumūlesthe \MSN}
} vṛkkahṛdayavahe ḍākinī. 
\pend

\pstart
doṣāvatīmahāviṣṭe kakṣastanavartinyau cakṣuḥpittavahe pukkasī.
mātarāsarvaryau \vedtext{nābhināsāgrasthe}{
	\variants{\MSN\PCreading ; nābhinā..kasāgrasthe \MSN}
} \vedtext{phupphusāntramālāvahe}{
	\variants{\emd ; phupphuṣāntranālāvahe \MSN}
} śabarī.
śītadoṣme mukhakhaṇṭhasthe pārśvatantūdaravahe caṇḍālī. 
pravaṇā hṛdaye viṣṭhāvahā, hṛṣṭavadanā\footnote{\begin{english}
	This \emph{nāḍī} has both the name \emph{hṛṣṭavadanā} and \emph{kṛṣṇavadanā} (the \emph{akṣara}s for \emph{hṛ} and \emph{kṛ} having similar forms in North Indian scripts).
	Here \TIB\ reads \emph{mdog nag ma} (read mdong? - TODO check mss) and therefore reflects the latter.
\end{english}} \vedtext{liṅge}{
	\variants{\emd ; hṛlliṅge \MSN ; \emph{Not reflected in Tib.}}
} sīmantamadhyagā ḍombī. 
\pend

% sīmāntamadhyagā - same phrasing used in MuĀ

% mi phyed ma dang phra gzugs ma ni spyi bor gnas pa ste/ grangs bzhin du sen mo dang/ so dang/ skra dang ba spu'i mtshan nyid rnams zung ngu 'bab pa ni rdo rje ma'o/ /
% gsal ma ni rna ba g-yas pa'i pags pa dang/ dri ma 'bab pa [D f. 222a] */ /dang/ g-yon ma ni rgyab kyi sgal tshigs kyi sha 'bab pa dag ni goo r'i'o/ /
% ya mi ni dang/ rus sbal skyes dag ni rna ba g-yon pa dang/ smin ma'i dbus na gnas pa ste/ rgyus pa dang/ rus pa dang/ dri 'bab ba dag ni chu ma'o/ /dngos po ma dang dbang bskur ma dag ni/ mig dang lag pa'i rtsa bar gnas te/ glo ba dang snying 'bab pa ste/ rdo rje mkha' 'gro ma'o/ /
% skyon ldan ma dang ma h'a ni ye dag ni mchan khung dang nu ma dag la gnas pa ste/ mig dang mkhris pa 'bab pa ste/ pukka s'i'o/ /
% ma mo dang mtshan mo dag ni lte ba dang sna rtse dag la gnas pa ste/ mchin pa dang rgyu ma dang dri 'bab pa sha pa r'i'o/ /
% bsil ster dang tsha ba ma dag ni kha dang mgrin par gnas te/ ngos kyi skud pa dang lto ba 'bab pa ste/ tsandal'i'o/ /
% phra ma na ni snying dang bshang ba 'bab pa'o/ /mdog nag ma ni mtshams mthar gnas pa ste/ dombi n'i'o/ /

\pstart
svarū\fnum{\MSN}{264}{r}piṇīsāmānye \vedtext{meḍhragudayoḥ}{
	\variants{\emd ; na | me<mayā>ḍhragudayoḥ}
} śleṣmapittavahe gaurī. 
hetudāyikāviyoge ūrujaṅghayoḥ śoṇitasvedavahe caurī. 
premaṇīsiddhe \vedtext{pādāṅguṣṭha}{
	\variants{\emd ; pādāṅguṣṭhali}
}pādapṛṣṭhayor \vedtext{medaḥkhedāśruvahe}{
	\variants{\conj ; medaḥkheṭavahe \MSN}
}\footnote{\begin{english}
	Here as the second member of the compound we expect a word meaning `tears'.
	Note that \TIB 's reading of \emph{mchin pa} is probably a scirbal error for \emph{mchi ma}.
	The conjecture `\emph{medo'śruvahe}' is also plausible, but \emph{medaḥkhedāśruvahe} is a more likely cause of error. 
	Kamalanātha, in his \emph{Ratnāvalī} (ms.\ f.\ 3r7), uses the word \emph{śokāśru} in this context, which can be regarded as an equivilant to \emph{khedāśru}.
\end{english}} vettālī. 
pāvakī\vedtext{sumane}{
	\variants{\emd ; samāne \MSN}
} aṅguṣṭhajānudvayasthe.
tatra pūrvā kheṭavahā, \vedtext{aparā}{
	\variants{\emd ; aparālā \MSN}
} siṃhāṇavahā.
te ime ghasmarī. 
\pend

% phra gzugs ma dang spyi ma dag ni mdoms dang bshang ba'i lam dang bad kan dang mkhris pa 'bab pa ste/ goo r'i'o/ /
% rgyu sbyin ma dang bral ba dag ni brla dang byin pa dag la gnas pa khrag dang rdul 'bab pa ste/ tsoo r'i'o/ /
% sdug ma dang dngos grub ma dag ni rkang pa'i sor mo dang rkang pa'i rgyab tu tshil dang mchin pa 'bab pa ste/ be t'a l'i'o/ /
% 'tshed pa dang/ yid bzangs ma dag ni/ mthe bo dang pus mo gnyis la gnas te/ de la snga ma ni bad kan 'bab pa/ phyi ma ni stobs dang snabs 'bab pa ste/ 'di ni ghasma r'i'o/ /

\pstart
hṛtkamalakarṇikāpūrvādidaleṣu yathākramaṃ trivṛttā\dsh kāminī\dsh gehā\dsh caṇḍikā\dsh māradārikāḥ. 
tatra \vedtext{prathamaṃ}{
	\variants{\emd ; prathamā \MSN}
} nāḍīdvayaṃ bhūcarī, \vedtext{[śeṣāḥ khecarī.]}{
	\lemma{śeṣāḥ khecarī}
	\variants{\diag\ (\TIB\ lhag ma rnams ni mkha' spyod ma'o/ /); \emph{deest} in \MSN}
}
atra ca yā nāḍī yaṃ \vedtext{prasūte}{
	\variants{\emd ; \hl{prasṛte} \MSN\ TODO check MuĀ mss}
} puṣṇāti gacchati vā sā tadvāhā.\footnote{\begin{english}
	This sentence is found in Ratnākaraśānti's MuĀ ad HeTa 1.1.14: \emph{tatra yā nāḍī yaṃ\footfoot{1} prasūte puṣṇāti gacchati vā sā tadvahā yathāyogam} (\footfoot{1}\emph{tatra yā nāḍī yaṃ}] ms-a ed; \emph{tatra nāḍī | nāḍī yaṃ} ms-b) (ms-a f.\ 17v; ms-c f.\ 13r; ed.\ p.\ 20).
	\TIB\ is problematic here: \emph{'di yang rtsa nas rab tu 'dzag pas/ rgyas par byed pa dang/ 'gro bar byed pa dang/ de nas cung zad 'bab pa'o/ /}.
	I am not certain what the translator intended by this formulation, but there appears to have been some confusion on his part.
	TODO: Discuss prasṛte v.\ prasūte.
	% MuĀ D 231v de la rtsa gang las gang skye zhing rgyas pa'am/ 'gro ba ji ltar rigs par de dang de las rgyu ba ste/ 
	% reading prasū confirmed in both MuĀ manuscripts, prasṛ confirmed in MSN
\end{english}} 
\pend

% * Tib. lhag ma rnams ni mkha' spyod ma'o/ /
% * Last sentence is from Muktāvalī ad 1.1.14 (ed. p. 20). Either misunderstood or translated in a strange way in Tib.

% snying ga'i padma'i lte ba'i shar la sogs pa'i 'dab ma rnams la go rims bzhin du gsum skor ma dang/ 'dod ma dang/ khyim ma dang/ gtum mo dang/ bdud sbyin ma rnams so/ /
% de la rtsa dang po gnyis ni sa spyod ma'o/ /
% lhag ma rnams ni mkha' spyod ma'o/ /
% 'di yang rtsa nas rab tu 'dzag pas/ rgyas par byed pa dang/ 'gro bar byed pa dang/ de nas cung zad 'bab pa'o/ /

\pstart
kiṃcetāḥ kāyavākcittadharmasambhoganirmāṇatribhava\vedtext{svabhāvāś}{
	\lemma{°svabhāvāś}
	\variants{\MSN\PCreading\ (°svabhāvāḥ |); °svabhāvāḥ | ś \MSN\ACreading}
}\footnote{\begin{english}
	The word \emph{tribhava} is not reflected in \TIB .
	See MuA on HeTa 1.1.3b on the correspondence of \emph{tribhava} with \emph{kāyavākcitta}.
	TODO: Quote MuA etc.
\end{english}} catuścakreṣu śarīreṣu vyavasthitāḥ. 
tatra nirmāṇacakraṃ \vedtext{viśvavarṇaṃ}{
	\variants{\conj ; viśvavarṇṇaṃ \MSN}
} catuḥṣaṣṭidalaṃ \vedtext{kṛṣṇa\dsh \begin{mantra}a\end{mantra}\dsh kārabījaṃ}{
	\variants{\emd ; kṛṣṇāṃkārabījaṃ \MSN}
} nābher adho vyavasthitam ūrdhvamukhaṃ ca.
dharmacakraṃ śuklāṣṭadalakamalaṃ kṣṇāvarṇa\dsh \begin{mantra}hūṁ\end{mantra}\dsh kārabījaṃ hṛddeśe \vedtext{vyavasthitam}{
	\variants{\emd ; vyavasthita \MSN}
}. 
kaṇṭhe sambhogacakraṃ \vedtext{raktaṣoḍaśadalaṃ}{
	\variants{\emd ; raktaṣoḍaśadala \MSN}
} raktapraṇavabījam.
śirasi śukladvātriṃśaddalaṃ śukla\dsh \begin{mantra}haṃ\end{mantra}\dsh kārabījam adhomukhaṃ kṣaratpīyūṣadhāraṃ mahāsukhacakram. 
atrānandakṣaṇabhedādivyavasthā gurūpadeśato bodhavyā.
\pend

% * tribhava missing from Tib.

% sku dang gsung dang thugs dang/ chos dang longs spyod rdzogs pa dang/ sprul pa'i ngo bo nyid ni 'khor lo bzhir lus la rnam par gnas pa'o/ /de la sprul pa'i 'khor lo ni kha dog sna tshogs pa padma 'dab ma drug cu rtsa bzhi pa'i lte ba'i 'og tu yi ge aa nag po steng du kha bltas pas rnam par gnas pa'o/ /chos kyi [D f. 222v] 'khor lo ni padma 'dab ma brgyad pa la yi ge h'um nag po snying gar rnam par gnas pa'o/ /
% mgrin par longs spyod kyi 'khor lo dmar po 'dab ma bcu drug pa la pra na ba sa bon dmar po'o/ /spyi bor 'dab ma sum cu rtsa gnyis pa dkar po la yi ge ham dkar po'i sa bon kha thur du bltas pa las bdud rtsi'i rgyun 'bab pa bde ba chen po'i 'khor lo'o/ /
% 'di dag dga' ba dang skad cig ma'i rnam par gzhag pa ni bla ma'i man ngag las rtogs par bya'o/ /

\pstart
\vedtext{anābhogayuganaddhādvayavāhi}{
	\variants{\conj\ (\TIB\ 'bad pa med par zung du 'jug pa gnyis med bar 'byung ba'i); anābhogayuganaddhavāhi}
} bodhicittasākṣātkaraṇaṃ\footnote{\begin{english}
	There are a few points to consider regarding the reading in \TIB : \emph{'bad pa med par zung du 'jug pa gnyis med bar 'byung ba'i byang chub kyi sems mngon du byed pa'i rgyu'o}.
	First, the translation evidently takes this as a complete sentence.
	It has perhaps confused \emph{karaṇa} with \emph{kāraṇa}.
	It also reflects the word \emph{advaya} within the compound ending \emph{vāhin}, and it connects this compound with the following word, thus qualifying \emph{bodhicitta}.
	These last two points are valid possibilities, and I wish to accept the former.
	We find a few parallels in Advayavajra's corpus for the compound \emph{yuganaddhādvayavāhi}: e.g., \emph{Amanasikārādhāra} (ed.\ p.\ 497), the \emph{Sekatātparyasaṅgraha} (ed.\ p.\ 413), and \emph{Pañcatathāgatamudrāvivaraṇa} (ed.\ p.\ 377). 

	Whether \emph{anābhogayuganaddhavāhi} should qualify \emph{bodhicitta} or \emph{sākṣātkaraṇa} is slightly more difficult to determine, but perhaps ultimately there is no great difference.
	\emph{Bodhicitta}, the innate nature of mind, is \emph{anābhogayuganaddhādvayavāhin} in that it supports (\emph{vāhin} in the sense of `bearing') the non-dual state of the effortless unity of bliss/compassion and emptiness; manifesting \emph{bodhicitta} is \emph{anābhogayuganaddhavāhin} in that it produces/leads to (\emph{vāhin} in the sense of \emph{pra-√sū} etc.) the non-dual state that is effortless coalescence.%
% 	`The natural stage is manifesting bodhicitta [in a way] that leads to effort-free coalescence.'
	% Mathes renders the sentence as follows: 'The natural completion stage is the act of making the \emph{bodhicitta}, which flows as the effortless inseparable union [of compassion and emptiness], directly manifest'.

	TODO: more details. See also Sekatātparyasaṅgraha ed. p. 411, without advaya.
\end{english}} svābhāvikaḥ kramaḥ. 
\pend

% 'bad pa med par zung du 'jug pa gnyis med bar 'byung ba'i byang chub kyi sems mngon du byed pa'i rgyu'o/ /ngo bo nyid kyi rim pa'o/ /

\pstart
tato bhāvanākhinno nairātmyāhaṃkāram u\fnum{\MSN}{264}{v}dvahan \vedtext{mantraṃ}{
	\variants{\emd ; mantra \MSN}
} japet.
tatrāmī sahajasiddhāḥ praṇavādyāḥ svāhāntāḥ pañcadaśasvarasvabhāvā \begin{mantra}a\end{mantra}\dsh kārādayo mantraḥ. tadyathā\Emdash \begin{mantra}oṁ a ā i ī u ū ṛ ṝ ḷ ḹ e ai o au aṁ svāhā\end{mantra}.
\pend

% de nas bsgom pas dub na bdag med ma'i nga rgyal dang ldan pas sngags bzlas par bya'o/ /de la 'dir lhan cig skyes par grub pa phra ba dang po h'um phat mthar gnas pa'i dbyangs bco lnga'i rang bzhin gyi aa la sogs pa'i sngags so/ /'di lta ste/ aom aa a'a ai a'i au a'u ri r'i li l'i ae aee ao aoo aam h'um phat sv'a h'a/ 

\pstart
\edtext{raktanairātmyāhaṃkāram}{
	\lemma{raktanairātmyāhaṃkāram \ldots\ tadanantaraṃ maṇḍalacakrasā°}
	\variants{\emph{The text beginning} raktanairātmyā° \emph{is written in a second hand. The text beginning} japet \emph{is written as a marginal addition. The addition ends} tadanantaraṃ maṇḍalacakrasā°.}
} udvahan [\vedtext{purakṣobhamantraṃ}{
	\variants{\diag\ (\TIB : grong khyer dkrug pa'i sngags); mantram imaṃ \MSN}
}] japet. tatrāyaṃ mantraḥ\Emdash \begin{mantra}oṁ a ka ca ṭa ta pa ya śa svāhā\end{mantra}.
\pend

% bdag med ma dmar mo'i nga rgyal dang ldan pas grong khyer dkrug pa'i sngags bzlas par bya'o/ /de la 'dir sngags ni/ aom aa ka tsa ta ta pa ya sha sv'a h'a/ zhes pa'o/ /

\pstart
\begin{mantra}oṁ akāro mukhaṃ sarvadharmāṇām ādyanutpannatvāt oṁ āḥ hūṁ phaṭ svāhā\end{mantra}\Emdash balimantraḥ. 
\pend

% aom aa k'a ro mu kham sarba dharm'a n'am/ a'a dya nutpanna tv'a ta/ aom a'a: h'um phat sv'a h'a/ zhes pa ni gtor ma'i sngags so/ /

\pstart
\begin{mantra}oṁ āḥ hūṁ\end{mantra}\Emdash samayādhiṣṭhānamantraḥ. 
\pend

% aom a'a: h'um zhes bya ba ni dam tshig byin gyis brlab pa'i sngags so/ /


\pstart
tadanantaraṃ maṇḍalacakrasākṣāddaśāyāṃ stanau hṛtvā \vedtext{svāṅgānāṃ}{
	\variants{\conj\ (\TIB : rang gi yan lag gi); svābhāṅganāṃ}
} kakkolamadhyavarti bolaṃ kuryāt. pārśvadvayaṃ ghaṇṭāṃ vidadhyāt. 
\pend

% de'i rjes su dkyil 'khor gyi 'khor lo mngon du byas pa'i dus su ni nu ma spangs pas rang gi yan lag gi kakkola'i nang du gnas pa'i bo lar bya'o/ /'gram pa gnyis ni dril bur 'gyur ba'o/ /

\medskip\versequote
\vedtext{[mantrayāne]}{
	\variants{\diag ; cakranayā \MSN ; gsang sngags tshul gyi \TIB}
} śāstra\vedtext{sāraṃ}{
	\variants{\emd ; \abbr sāraṃ raṃ \MSN}
}\footnote{\begin{english}
	I have hesitantly settled on this conjecture with some inspiration from \TIB : \emph{gsang sngags tshul gyi bstan bcos snying/ /bla ma dam pa'i 'bad pa las shes pa}.
	We might expect \emph{mantranaya} for \emph{gsang sngags thsul}, but that would be metrically bad; a genitive case ending would also be impossible, but perhaps we could also conjecture the compound \emph{mantrayānaśāstrasāraṃ}.
	My solution has the slight disadvantage of forming a \emph{ra-vipulā} in the \emph{pāda}, which is unexpected but not impossible, and we can't easily explain the formation of the corrupt manuscript reading: \emph{cakranayā śāstrasāraṃ raṃ}.
	Note that \TIB\ has also interpreted \emph{guroḥ} as a genitive form connected to \emph{yatnena}, leading a bizarre meaning: `Having understood the essence of the \emph{śāstra}s of the Way of Mantra by means of [my] Guru's effort'.
	I interpret the text as I have construed it as follows: `Having diligently understood from [my] guru the essence of \emph{śāstra} in the Vehicle of Mantra \ldots '	
\end{english}} jñātvā yatnena sadguroḥ | &
kṛpayā vihito 'smābhir nairātmyāyāḥ prakāśakaḥ || \&

% * cakranayā and gsangs sngags tshul gyi are both strange. Conjecture mantrāmnayaṃ perhaps?

% gsang sngags tshul gyi bstan bcos snying/ /
% bla ma dam pa'i 'bad pa las shes pa/ /
% bdag med ma ni rab tu gsal ba ni/ /
% bdag gi snying rje yis ni sems kyis byas/ /

\medskip\versequote
gahanamaṇḍalacakraviniścayo bata bhavet katham atra śarīriṇām | &
śabaranāthapadāmbujareṇubhir yadi na \vedtext{rūkṣitamastakatā}{
	\variants{\conj ; rūkṣitamastako \MSN}
} bhavet ||\footnote{\begin{english}
	I have made this conjecture by modifying the transmitted text minimally to arrive at something (perhaps) coherent.
	\TIB\ appears to be translated rather freely: \emph{%
		dkyil 'khor 'khor lo zab mo yi/ /
	 	rnam par nges par 'gyur ba ni/ /
		ji ltar 'dir ni lus can rnams/ /
		ri khrod mgon gyi zhabs kyi chu skyes kyi/ /
		rdul rnams spyi bos ma blangs pa/ /
		de dag nges par ji ltar 'gyur/ /%
	} We can see evidence for \emph{mastaka} with the word \emph{spyi bo}.
	The final \emph{pāda}, \emph{de dag nges par ji ltar 'gyur}, does not yield a coherent meaning for me.
	The metre is Drutavilambita.
\end{english}}\&

% (Drutavilambitam)
% 
% LLLGLLGLLGLG LLL GLL    GLL    GLG
% LLLGLLGLLGLG LLL GLL    GL     GLG
%   

% dkyil 'khor 'khor lo zab mo yi/ /
% rnam par nges par 'gyur ba ni/ /
% ji ltar 'dir ni lus can rnams/ /
% ri khrod mgon gyi zhabs kyi chu skyes kyi/ /
% rdul rnams spyi bos ma blangs pa/ /
% de dag nges par ji ltar 'gyur/ /

\medskip\versequote
\vedtext{[abhisamayasuvistṛtau]}{
	\variants{\diag ; abhisamayavistarite \MSN ; abhisamayasuvistare \emph{possible emd.}}
} yad āptaṃ &
\hspace{20pt} kuśalam \vedtext{anena bhavet}{
	\variants{\conj\ (\textsc{Isaacson}); anena \MSN)}
} samastalokaḥ | \&
\versequote
kuliśadharapadapra\vedtext{tiṣṭhitātmā}{
	\lemma{°tiṣṭhitātmā}
	\variants{\emd ; tiṣṭhitātnā \MSN}
} &
\hspace{20pt} hatabhuvana\vedtext{traya}{
	\lemma{°traya°}
	\variants{\emd ; °trayaṃ° \MSN}
}duḥkhadaurmmanasyaḥ ||\footnote{\begin{english}
	I have yet to find a compelling conjecture for the first \emph{pāda}.
	\TIB\ reads as follows: \emph{%
		mngon par rtogs pa 'di yis thob pa yis/ /
		dge ba 'di yis 'jig rten mtha' dag ni/ /
		srid pa gsum gyi sdug bsngal yid mi bde spangs te/ / 
		rdo rje 'dzin pa'i go 'phang rab gnas shog /
	}. The metre is Puṣpitāgrā.
\end{english}} \&

% puṣpitāgrā metre
%  abhisamayavi       vistare         yad āptam
% LLLLLLGLG                       LGG
% LLLLGLLGLGLGL
% LLLLLLGLGLGG
% LLLLGLLGLGLGG


% mngon par rtogs pa 'di yis thob pa yis/ /
% dge ba 'di yis 'jig rten mtha' dag ni/ /
% srid pa gsum gyi sdug bsngal yid mi bde spangs te/ / [D f. 223r]*/ /
% rdo rje 'dzin pa'i go 'phang rab gnas shog /

\medskip\pstart
nairātmyāprakāśaḥ samāptaḥ. 
\pend

\bigskip\pstart
\begin{center}
|| kṛtir iyaṃ śrīmatpaṇḍitācāryāvadhūtādvayavajrapādānām iti ||
\end{center}
\pend

% dpal pandi ta gnyis su med pa'i rdo rjes mdzad pa'i bdag med ma'i sgrub thabs rdzogs so// //rgya gar gyi mkhan po badzra p'a ni las mnyan te slad kyi dge slong mtshur dzny'a na a'a ka ras bsgyur ba'o// // 


\end{sanskrit}
\endnumbering
\end{document}
