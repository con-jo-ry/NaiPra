\documentclass[12pt,twoside]{article}
\usepackage{subfiles} 						% To import each section
\usepackage[osf,p]{libertinus}
\usepackage{libertine}
\usepackage{microtype}
\usepackage[pdfusetitle,hidelinks]{hyperref}
\usepackage{endnotes} % To move \footnote{} to an endnote.
\let\footnote=\endnote
\renewcommand{\theendnote}{\roman{endnote}}
\usepackage{color,soul}
\usepackage{babel}
\usepackage{setspace}
\usepackage[series={A},noend,noeledsec,noledgroup]{reledmac}

\usepackage{verbatim}
\Xarrangement[A]{paragraph}
\usepackage{fancyhdr}

\usepackage[nospacearound]{extdash} % in order to allow hyphanation with dashes
\def\EXD@emdash{\hbox{—}}

% headings in apparatus:
\newcommand{\appnote}{\noindent \hrulefill\ {\footnotesize Notes} \hrulefill \vskip.3em}
\let\footnoterule=\appnote
\newcommand{\appvariants}{\hrulefill\ {\footnotesize Variants} \hrulefill \vskip.3em}
\let\Bfootnoterule=\appvariants

\usepackage{fontspec}
\usepackage{polyglossia} 		
	\setmainlanguage{english}
	\setotherlanguage{sanskrit}	
	\newfontfamily\sanskritfont[Ligatures=TeX]{Linux Libertine}
	\newfontfamily\mantrafont[Ligatures=TeX]{STIX Two Text}[Scale=MatchLowercase]
	\newfontfamily\devfont[Script=Devanagari]{Noto Sans Devanagari}

\usepackage{longtable} 			% For sigla and symbols

%% page setup %%
\renewcommand*{\numlabfont}{\normalfont\small}
\linenummargin{inner}
\lineation{page}
\linespread{1.1}


% for quoting verse
\newcommand{\versequote} {
	\setlength{\stanzaindentbase}{25pt}
	\setstanzaindents{1,1,1}
	\stanza
}

% make brackets etc. never in italics
% must be turned off before printing bibliography
\usepackage[biblatex]{embrac}
\providecommand\textsi[1]{#1} % To avoid error in xelatex looking for textsi command
\AddOpEmph{|}
\AddOpEmph{/}


\newenvironment{prosequote} {\medskip\begingroup\leftskip3em\rightskip\leftskip\noindent}{\par\endgroup\medskip}

% environment for putting mantras in smallcaps
\newenvironment{mantra}{\begin{mantrafont}\scshape}{\end{mantrafont}}

% Command to make footnotes in footnotes
\newcommand{\footfoot}[1] {\textsuperscript{\textnormal{#1}}}

%sigla
\newcommand{\PCreading}{$^{pc}$}
\newcommand{\ACreading}{$^{ac}$}
\newcommand{\MSN}{Ṅ}
\newcommand{\TIB}{T\textsubscript{D}}
\newcommand{\sigmareading}[1]{$\Sigma$\textsubscript{#1}}

%folio and page numbers
\newcommand{\fnum}[3]{\hspace{0em}$\makebox[0pt][c]{\raisebox{-1.4ex}{\tiny$\perp$}}$\ledouternote{\textenglish{\textnormal{#1\space fol.\space #2}$^{#3}$}}}
\newcommand{\pnum}[2]{\hspace{0em}$\makebox[0pt][c]{\raisebox{-1.4ex}{\tiny$\perp$}}$\ledouternote{\textenglish{\textnormal{#1\space p.\space #2}}}}


% apparatus
\newcommandx{\variants}[2][1,usedefault]{\Afootnote[#1]{#2}}


% main edit commands
\newcommand{\varbrace}[2]{$\makebox[0pt][#1]{\raisebox{-0.8ex}{\footnotesize$#2$}}$}
\newcommand{\lacbrace}[2]{$\makebox[0pt][#1]{\raisebox{0.8ex}{\footnotesize$#2$}}$}
\newcommand{\vedtext}[2]{\hspace{0em}\varbrace{l}{\llcorner}\edtext{#1}{#2}\varbrace{r}{\lrcorner}\hspace{0em}}
\newcommand{\ledtext}[2]{\hspace{0em}\lacbrace{l}{\ulcorner}\edtext{#1}{#2}\lacbrace{r}{\urcorner}\hspace{0em}}

% Symbols and sigla
\newcommand{\abbr} {°} 						% abbr for abbreviation 
\newcommand{\stho} {○} 						% stho for string hole
\newcommand{\dam} {×} 						% dam for damaged
\newcommand{\emd} {\emph{em.}}
\newcommand{\conj} {\emph{conj.}}
\newcommand{\diag} {\emph{diag.\ conj.}}
\newcommand{\possibleemd} {\emph{possible em.}}
\newcommand{\possibleconj} {\emph{possible conj.}}
\newcommand{\crux} {\hspace{0em}\textsuperscript{†}\hspace{0em}}
\newcommand{\dsh} {\hspace{0em}-\hspace{0em}}

%%% Bibliography
\usepackage[authordate16,backend=biber]{biblatex-chicago}
\addbibresource{~/Documents/personal/bibliography.bibtex}
\renewcommand{\mkbibnamefamily}[1]{\textsc{#1}}
\renewcommand\postnotedelim{\addcolon\addspace} % adds colon after year in citation.

\DeclareBibliographyCategory{fullcited}
\newcommand{\mybibexclude}[1]{\addtocategory{fullcited}{#1}}

\title{Nairātmyāprakāśa}
\author{Advayavajra \\ ed.\ Ryan Conlon}

\begin{document}
	\maketitle

% For page numbers
\pagestyle{fancy}
\fancyhf{}
\fancyhead[LE,RO]{\thepage}
\fancyhead[RE]{Advayavajra's \emph{Nairātmyāprakāśa}}
\fancyhead[LO]{\leftmark\ - DRAFT EDITION by Ryan Conlon}

	\section{Sigla and Symbols}	
	\subfile{naipra_sigla.tex}

	\section{Edition of the Sanskrit Text}
	\subfile{naipra_ed.tex}

	\section{Collation of the Tibetan Text}
	\subfile{naipra_tib.tex}

	\section{English Translation}
	\subfile{naipra_trans.tex}

	\newpage
	\theendnotes

	\EmbracOff

	\section*{Primary Sources}
% remove space before longtable
\setlength{\LTpre}{0pt}
\setlength{\LTpost}{0pt}

\noindent\emph{Amanasikārādhāra} by Advayavajra
\noindent\begin{longtable}{ p{0.07\textwidth} p{0.93\textwidth} }
	& \cite[489–498]{mathes2015} 
\end{longtable}

\noindent\emph{Dveṣavajrasādhana} by Bhadrapāda
\noindent\begin{longtable}{ p{0.07\textwidth} p{0.93\textwidth} }
	& \cite[vol.\ 2 pp.\ 335–360]{gerloff2020} 
\end{longtable}

\noindent\emph{Pañcatathāgatamudrāvivaraṇa} by Advayavajra
\noindent\begin{longtable}{ p{0.07\textwidth} p{0.93\textwidth} }
	& \cite[371–384]{mathes2015} 
\end{longtable}

\noindent\emph{Bhramaharasādhana} by Ratnākaraśānti
\noindent\begin{longtable}{ p{0.07\textwidth} p{0.93\textwidth} }
	& \fullcite*{isaacson2002}.\mybibexclude{isaacson2002} 
\end{longtable}

\noindent\emph{Nāmamantrārthāvalokinī} by Vilāsavajra
\noindent\begin{longtable}{ p{0.07\textwidth} p{0.93\textwidth} }
	& See \textcite{tribe2016}.
\end{longtable}

\noindent\emph{Muktāvalī} (MuĀ) by Ratnākaraśānti
\noindent\begin{longtable}{ p{0.07\textwidth} p{0.93\textwidth} }
	ed. & \fullcite*{tripathi2001}.\mybibexclude{tripathi2001} \\
	ms-a & National Archives Kathmandu, 4/19 (NGMCP A 994/6). Palm-leaf, Bhuṃjimol script, 115 folios.\\
	ms-b & NGMCP E 260/2. Palm-leaf, Proto-Bengali script, 25 follios. \\
	ms-c & Tokyo University Library, M.\ 513. Palm-leaf, Proto-Bengali script, 40 folios. (part of E 260-2) \\
	D & dPal dgyes pa'i rdo rje'i dka' 'grel mu tig phreng ba by Ratnākaraśānti. Translation by Śāntibhadra and 'Gos lhas btsas. In sDe dge bstan 'gyur, Tōh.\ 1189, rgyud, vol.\ 4 (ga), fo.\ 221r1–297r7. 
\end{longtable}

\noindent\emph{Yogāmbaratantra}
\noindent\begin{longtable}{ p{0.07\textwidth} p{0.93\textwidth} }
	& National Archives Kathmandu, 4/2917 (NGMCP A 142-12). Paper (?), unkown script, Nepal Saṃvat 1036, 48 folios.
\end{longtable}

\noindent\emph{Ratnāvalī} by Kamalanātha 
\noindent\begin{longtable}{ p{0.07\textwidth} p{0.93\textwidth} }
	& \emph{Ratnāvalī Hevajrapañjikā}. Kaiser Library, 231 (NGMPP C 26-4[2]). Palm leaf, proto-Bengali script, 23 fos., complete. 
\end{longtable}

\noindent\emph{Lalitavistara}
\noindent\begin{longtable}{ p{0.07\textwidth} p{0.93\textwidth} }
	ed. & \fullcite*{lefmann1902}.\mybibexclude{lefmann1902} \\ 
	D & \emph{'Phags pa rgya cher rol pa zhes bya ba theg pa chen po'i mdo}. Tōh.\ 95, mdo sde, vol.\ 46 (Kha) 1v1-216v7. Translation by Jinamitra, Dānaśīla, Munivarman, and Ye shes sde.
\end{longtable}

\noindent\emph{Vajravārāhīsādhana} by Advayavajra
\noindent\begin{longtable}{ p{0.07\textwidth} p{0.93\textwidth} }
	ed-f & In \fullcite*{finot1934}.\mybibexclude{finot1934} \\ 
	ed-b & \emph{Sādhanamālā} no.\ 217.
	Dt & \emph{rDo rje phag mo sgrub pa'i thabs} by gNyis su med pa'i rdo rje. Trans.\ by Vajrapāṇi and mTshur ston dBang nges. sDe dge bstan 'gyur, Tōh.\ 1542, rgyud, vol.\ 23 (Za), fols.\ 181r1-182r7.
	Dy & \emph{rDo rje phag mo'i sgrub thabs} by gNyis med rdo rje. Trans.\ by Yar lung Grags pa rgyal mtshan. sDe dge bstan 'gyur, Tōh.\ 3607, rgyud, vol.\ 77 (Mu), fol.\ 229v2–230v5.
\end{longtable}

\noindent\emph{Vajravārāhīsādhana} by Umpāpatideva
\noindent\begin{longtable}{ p{0.07\textwidth} p{0.93\textwidth} }
	ed-f & In \fullcite*{english2002} \\ 
\end{longtable}

\noindent\emph{Saptākṣarasādhana} 
\noindent\begin{longtable}{ p{0.07\textwidth} p{0.93\textwidth} }
	ed.\ & \emph{Sādhanamālā} no.\ 251. \\
	D & \emph{Yi ge bdun pa'i sgrub thabs} by gNyis su med pa'i rdo rje. Trans.\ by Mar pa chos kyi dbang phyug. sDe dge bstan 'gyur, Tōh.\ 1483, rgyud, vol.\ 22 (Zha), fols.\ 130r7–133v1. 
\end{longtable}

\noindent\emph{Sādhanamālā} 
\noindent\begin{longtable}{ p{0.07\textwidth} p{0.93\textwidth} }
	& \fullcite*{bhattacharyya1925}.\mybibexclude{bhattacharyya1925}.
\end{longtable}

\noindent\emph{Sādhanavidhāna} (a.k.a.\ \emph{Buddhagranthasaṅgraha}) 
\noindent\begin{longtable}{ p{0.07\textwidth} p{0.93\textwidth} }
	& National Archives Kathmandu, NAK 3-693 (NGMPP A 936/11). Palm leaf, hooked Newar script, dated 658 Nepal Saṃvat, 109 folios. 
\end{longtable}

\noindent\emph{Sekatātparyasaṅgraha} by Advayavajra
\noindent\begin{longtable}{ p{0.07\textwidth} p{0.93\textwidth} }
	& \cite[403–414]{mathes2015} 
\end{longtable}

\noindent\emph{Śrīcakrasaṃvaropadeśa} by Advayavajra
\noindent\begin{longtable}{ p{0.07\textwidth} p{0.93\textwidth} }
	D & \emph{dPal 'khor lo sdom pa'i man ngag} by gNyis med rdo rje. Trans.\ by Varendraruci and rMa ban Chos 'bar. sDe dge bstan 'gyur, Tōh.\ 1485, rgyud, vol.\ 22 (Zha), fols.\ 139r1–143v7. 
\end{longtable}

\noindent\emph{Hevajratantra} (HeTa) 
\noindent\begin{longtable}{ p{0.07\textwidth} p{0.93\textwidth} }
	ed. & \fullcite*{snellgrove1959}.\mybibexclude{snellgrove1959} \\
	N\textsubscript{a} & National Archives Kathmandu 7/11 (NGMPP A 993/7). Palm leaf, 45 ff., incomplete. \\
	N\textsubscript{b} & National Archives Kathmandu, 5/93 (NGMPP A 48-8). Palm leaf, 28 ff., damaged. \\
	C & Cambridge University Library, MS Add.1697.2. Palm leaf, Bengali script, 34 fos., incomplete. \\
	P & Manuscript images of unknown provenance. Palm leaf, Newar script, incomplete. \\
	K & Kaiser Library KL 126 (NGMPP C 14-4), paper, Newari, 52 folios, complete. (775 NS = 1654 CE). 
\end{longtable}

\noindent\emph{Hevajraviśuddhinidhi} by Advayavajra
\noindent\begin{longtable}{ p{0.07\textwidth} p{0.93\textwidth} }
	ms & In \emph{Hevajrasādhanasaṅgraha}, fols.\ 65r-80v. \\
	D & \emph{Kye rdo rje'i sgrub pa'i thabs rnam par dag pa'i gter} by gNyis su med pa'i rdo rje. Trans.\ by Vanaratna and 'Gos gzhon nu dpal. sDe dge bstan 'gyur, Tōh.\ 1244, rgyud, vol.\ 9 (Nya), fols.\ 175r1-189r4. 
\end{longtable}

\noindent\emph{Hevajrasādhanasaṅgraha}
\noindent\begin{longtable}{ p{0.07\textwidth} p{0.93\textwidth} }
	ms & Reproductions of photographs made by Rāhula Sāṅkṛtyāyana at Ngor monastery. Niedersächsische Staats- und Universitätsbibliothek Göttingen, Xc 14/38. 
\end{longtable}

\noindent\emph{Hevajrasādhanopāyikā} by Saroruhavajra
\noindent\begin{longtable}{ p{0.07\textwidth} p{0.93\textwidth} }
	& \textcite*[vo.\ 1 pp.\ 99–121]{gerloff2020} 
\end{longtable}

\noindent\emph{*Hevajrasādhana} by Advayavajra
\noindent\begin{longtable}{ p{0.07\textwidth} p{0.93\textwidth} }
	& \emph{Kye rdo rje zhes bya ba'i sgrub pa'i thabs} by gNyis su med pa'i rdo rje. No translator/paṇḍita attribution.  In sDe dge bstan 'gyur, Tōh.\ 1243, rgyud, vol.\ 9 (Nya), fo.\ 162r3-175r1. 
\end{longtable}

\noindent\emph{Hevajrākhya} by Advayavajra
\noindent\begin{longtable}{ p{0.07\textwidth} p{0.93\textwidth} }
	& In \emph{Hevajrasādhanasaṅgraha}, fo.\ 8r-22v.
\end{longtable}

	\section*{Secondary Sources}
	\printbibliography[notcategory=fullcited,resetnumbers,heading=none]

\end{document}


% MS citations: material, akṣara, date, folios
% Tibetan canon: Tibetan name by author (as given in colophon). Trans.\ by X. Canon, number, section, vol. X (letter), fols. 1r1–1r2.
